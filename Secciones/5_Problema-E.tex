\section{Pregunta \texttt{e)}}\label{pregunta-e}
\subsection{Desarrollo}

Se quiere ahora determinar los valores de $\mora{k_{c}}$ para los cuales el
sistema se comporta de primer y segundo orden estable. Se sabe que para que
esto ocurra, la razón entre las partes reales de las raíces debe ser igual a
$10$.

Se empieza entonces por obtener la función de transferencia de lazo,
\begin{align}
    l(s) &= \nara{k_{st}k_{a}} \frac{\rojo{\psi}(s)}{\verd{v_{i}}(s)} \\
    l(s) &= \frac{3033s + 13531}{s^{3} + 2.83s^{2} + 80.87s + 119.5}
\end{align}

Luego, para obtener los valores de las raíces en L.C, se busca resolver la
ecuación:
\begin{gather}
    1 + \mora{k_{c}}l(s) = 0 \\
    1 + \mora{k_{c}}\frac{3033s + 13531}{s^{3} + 2.83s^{2} + 80.87s + 119.5} = 0
\end{gather}

Luego, sean $\lambda_{i}$ los polos del sistema, entonces se puede reescribir
esta expresión como,
\begin{equation}
    (s - \lambda_{0})(s - \lambda_{1})(s - \lambda_{2}) = 0
\end{equation}

Donde $\lambda_{i}$ dependerá del valor de $\mora{k_{c}}$. Entonces, para que
el sistema se comporte como uno de primer o segundo orden, se necesita que
$\mathfrak{Re}(\lambda_{0}) = 10\mathfrak{Re}(\lambda{1})$. Se busca entonces
las soluciones para esta ecuación, y para esto la capacidad de cálculo simbólico
de \textit{MATLAB} fue utilizada, como se observa en el \autoref{lst:preg-e}.
\begin{align}
    &\boxed{\mora{k_{c}} = -0.0083} & &\boxed{\mora{k_{c}} = 0.0108}
\end{align}

Este código otorga que el valor de $\mora{k_{c}}$ para el cual el sistema se
comporta como de primer orden es $\mora{k_{c}} = -0.0083$, mientras que para
que este se comporte como de segundo orden es $\mora{k_{c}} = 0.0108$.

\begin{figure}[ht]
  \centering
  % This file was created by matlab2tikz.
%
%The latest updates can be retrieved from
%  http://www.mathworks.com/matlabcentral/fileexchange/22022-matlab2tikz-matlab2tikz
%where you can also make suggestions and rate matlab2tikz.
%
\begin{tikzpicture}

\begin{axis}[%
width=5.958cm,
height=6cm,
at={(0cm,0cm)},
scale only axis,
separate axis lines,
every outer x axis line/.append style={white!15!black},
every x tick label/.append style={font=\color{white!15!black}},
every x tick/.append style={white!15!black},
xmin=-12,
xmax=2,
every outer y axis line/.append style={white!15!black},
every y tick label/.append style={font=\color{white!15!black}},
every y tick/.append style={white!15!black},
ymin=-30,
ymax=30,
axis background/.style={fill=white},
legend pos=outer north east
]
\addplot [color=gray, dotted, forget plot]
  table[]{Diagramas/data/lgr_c_neg-1.tsv};
\addplot [color=gray, dotted, forget plot]
  table[]{Diagramas/data/lgr_c_neg-2.tsv};

\addplot [color=red, line width=1.5pt]
  table[]{Diagramas/data/lgr_c_pos-3.tsv};
  \addlegendentry{$\mora{k_{c}} > 0$}
\addplot [color=ForestGreen, line width=1.5pt, forget plot]
  table[]{Diagramas/data/lgr_c_pos-4.tsv};
\addplot [color=cyan, line width=1.5pt, forget plot]
  table[]{Diagramas/data/lgr_c_pos-5.tsv};

\addplot [color=red, dashed, line width=1.5pt]
  table[]{Diagramas/data/lgr_c_neg-3.tsv};
  \addlegendentry{$\mora{k_{c}} < 0$}
\addplot [color=ForestGreen, dashed, line width=1.5pt, forget plot]
  table[]{Diagramas/data/lgr_c_neg-4.tsv};
\addplot [color=cyan, dashed, line width=1.5pt, forget plot]
  table[]{Diagramas/data/lgr_c_neg-5.tsv};

\addplot[only marks, mark=o, mark size=2.0pt, thick, draw=Fuchsia, forget plot] table[]{Diagramas/data/lgr_c_pos-6.tsv};
\addplot[only marks, mark=x, mark size=2.5pt, thick, draw=Fuchsia, forget plot] table[]{Diagramas/data/lgr_c_pos-7.tsv};

\addplot[only marks, mark=square, mark size=2.0pt, thick, draw=Gray] table[row sep=crcr]{%
  -1.3498 +7.3193\\
  -1.3498 -7.3193\\
  -0.1350 0.0000\\
  };
  \addlegendentry{Polos en L.D. para $\mora{k_{c}}=-8.3\unit{m}$}

\addplot[only marks, mark=square, mark size=2.0pt, thick, draw=Lavender] table[row sep=crcr]{%
  -0.2362 +10.6078\\
  -0.2362 -10.6078\\
  -2.3622 0.0000\\
  };
  \addlegendentry{Polos en L.D. para $\mora{k_{c}}=10.8\unit{m}$}
\end{axis}
\end{tikzpicture}%

  \caption{Lugar geométrico de las raíces para el sistema con controlador proporcional.}
  \label{fig:lgr-e}
\end{figure}

Se grafica nuevamente el L.G.R. y se observa entonces que para las ganancias dadas,
se tienen los siguientes polos:
\begin{multicols}{2}
    Polos para $\mora{k_{c}} = -8.3\unit{m}$
    \begin{itemize}
        \item $\lambda_{0} = -1.3498 + 7.3193j$
        \item $\lambda_{1} = -0.1350$
        \item $\lambda_{2} = -1.3498 - 7.3193j$
    \end{itemize}

    Polos para $\mora{k_{c}} = 10.8\unit{m}$
    \begin{itemize}
        \item $\lambda_{0} = -2.3622$
        \item $\lambda_{1} = -0.2362 +10.6078j$
        \item $\lambda_{2} = -0.2362 -10.6078j$
    \end{itemize}
\end{multicols}

\FloatBarrier
\subsection{Comentarios}

De lo obtenido se puede verificar entonces que para ambos casos, se cumple que
$\mathfrak{Re}(\lambda_{0}) = 10\mathfrak{Re}(\lambda{1})$. Además, cuando
$\mora{k_{c}} = -8.3\unit{m}$, los polos complejos conjugados son aquellos que
tendrán bajo aporte en la dinámica del sistema, haciendo bien de este caso aquel
que se comporta como un sistema de primer orden, con polo dominante $\lambda_{1}$,
mientras que cuando $\mora{k_{c}} = 10.8\unit{m}$, los polos dominantes son
$\lambda_{1,2}$, y entonces $\lambda{0}$ no tendrá mayor aporte a la dinámica, por
lo que el sistema se comporta como uno de segundo orden.

Es importante mencionar además que esos valores de $\mora{k_{c}}$ no son los
únicos que permiten hacer esta simplificación, y que en realidad existe una
infinidad de ellos, ya que con tal que exista un polo cuya parte real sea mucho
mayor que la parte real de el resto, entonces el sistema puede comportarse como
uno de menor orden. Para este caso en específico, todas aquellas ganancias que
permiten que la condición $\mathfrak{Re}(\lambda_{0}) \geq 10\mathfrak{Re}(\lambda{1})$
sea verdadera, permiten que el sistema tenga esta simplificación. Así, las
ganancias determinadas son solamente los menores valores para los cuales esto
se permite.
