\section{Pregunta \texttt{f)}}\label{pregunta-f}

\subsection{Desarrollo}

A continuación, queremos determinar los valores de $\mora{k_{c}}$ para los cuales
el sistema se hace marginalmente estable. Es decir, buscamos para qué valores de
$\mora{k_{c}}$ existe por lo menos un polo del sistema cuya parte real sea $0$.

Podemos observar del L.G.R. (\autoref{fig:lgr-f}) que existirá un valor positivo
de $\mora{k_{c}}$ para el cual el sistema tendrá dos polos complejos conjugados
con parte real $0$, mientras que para valores negativos de $\mora{k_{c}}$, habrá
un solo valor que hará que el sistema tenga un polo real en el origen.

Entonces, utilizamos un método iterativo (véase \autoref{lst:preg-f}) con un
algoritmo de fuerza bruta para encontrar los valores de $\mora{k_{c}}$ que
cumplen esta condición, y obtenemos que estos son,
\begin{equation}
    \boxed{\mora{k_{c}} = 0.0222} \quad \text{y} \quad \boxed{\mora{k_{c}} = -0.0088}
\end{equation}

\begin{figure}[ht]
  \centering
  % This file was created by matlab2tikz.
%
%The latest updates can be retrieved from
%  http://www.mathworks.com/matlabcentral/fileexchange/22022-matlab2tikz-matlab2tikz
%where you can also make suggestions and rate matlab2tikz.
%
\begin{tikzpicture}

\begin{axis}[%
width=5.958cm,
height=6cm,
at={(0cm,0cm)},
scale only axis,
separate axis lines,
every outer x axis line/.append style={white!15!black},
every x tick label/.append style={font=\color{white!15!black}},
every x tick/.append style={white!15!black},
xmin=-12,
xmax=2,
every outer y axis line/.append style={white!15!black},
every y tick label/.append style={font=\color{white!15!black}},
every y tick/.append style={white!15!black},
ymin=-30,
ymax=30,
axis background/.style={fill=white},
legend pos=outer north east
]
\addplot [color=gray, dotted, forget plot]
  table[]{Diagramas/data/lgr_c_neg-1.tsv};
\addplot [color=gray, dotted, forget plot]
  table[]{Diagramas/data/lgr_c_neg-2.tsv};

\addplot [color=red, line width=1.5pt]
  table[]{Diagramas/data/lgr_c_pos-3.tsv};
  \addlegendentry{$\mora{k_{c}} > 0$}
\addplot [color=ForestGreen, line width=1.5pt, forget plot]
  table[]{Diagramas/data/lgr_c_pos-4.tsv};
\addplot [color=cyan, line width=1.5pt, forget plot]
  table[]{Diagramas/data/lgr_c_pos-5.tsv};

\addplot [color=red, dashed, line width=1.5pt]
  table[]{Diagramas/data/lgr_c_neg-3.tsv};
  \addlegendentry{$\mora{k_{c}} < 0$}
\addplot [color=ForestGreen, dashed, line width=1.5pt, forget plot]
  table[]{Diagramas/data/lgr_c_neg-4.tsv};
\addplot [color=cyan, dashed, line width=1.5pt, forget plot]
  table[]{Diagramas/data/lgr_c_neg-5.tsv};

\addplot[only marks, mark=o, mark size=2.0pt, thick, draw=Fuchsia, forget plot] table[]{Diagramas/data/lgr_c_pos-6.tsv};
\addplot[only marks, mark=x, mark size=2.5pt, thick, draw=Fuchsia, forget plot] table[]{Diagramas/data/lgr_c_pos-7.tsv};

\addplot[only marks, mark=square, mark size=2.0pt, thick, draw=Gray] table[row sep=crcr]{%
   0       12.1792\\
   0      -12.1792\\
  -2.8345   0\\
  };
  \addlegendentry{Polos en L.C. para $\mora{k_{c}}=22.2\unit{m}$}

\addplot[only marks, mark=square, mark size=2.0pt, thick, draw=Lavender] table[row sep=crcr]{%
  -1.4173  7.2161\\
  -1.4173 -7.2161\\
   0        0\\
  };
  \addlegendentry{Polos en L.C. para $\mora{k_{c}}=-8.8\unit{m}$}
\end{axis}
\end{tikzpicture}%

  \caption{Lugar geométrico de las raíces para el sistema con controlador proporcional.}
  \label{fig:lgr-f}
\end{figure}

Luego, determinamos los polos del sistema para estas ganancias, y observamos que,
\begin{multicols}{2}
    Polos para $\mora{k_{c}} = 22.2\unit{m}$
    \begin{itemize}
        \item $\lambda_{0} = -2.8345$
        \item $\lambda_{1} = +12.1792j$
        \item $\lambda_{2} = -12.1792j$
    \end{itemize}

    Polos para $\mora{k_{c}} = -8.8\unit{m}$
    \begin{itemize}
        \item $\lambda_{0} = 0$
        \item $\lambda_{1} = -1.4173 + 7.2161j$
        \item $\lambda_{2} = -1.4173 - 7.2161j$
    \end{itemize}
\end{multicols}

\subsection{Comentarios}

De los valores de ganancia obtenidos, podemos comentar que, si bien el sistema
es marginalmente estable para $\mora{k_{c}} = -8.8\unit{m}$, la oscilación de
este no será sostenida, puesto que el polo que hace al sistema marginalmente
estable no tiene componente imaginaria. Entonces, la oscilación del sistema
estará dada únicamente por los polos $\lambda_{1,2}$.

Por otro lado, cuando la ganancia es de $\mora{k_{c}} = 22.2\unit{m}$, el
sistema sí mantendrá una oscilación sostenida, y lo hará a una frecuencia de
$\frac{\mathfrak{Im}(\lambda_{1})}{2\pi} = 1.94\ \unit{Hz}$.
