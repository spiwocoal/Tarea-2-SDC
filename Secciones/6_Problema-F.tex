\section{Pregunta \texttt{f)}}\label{pregunta-f}

\subsection{Desarrollo}

A continuación, queremos determinar los valores de $\mora{k_{c}}$ para los cuales
el sistema se hace marginalmente estable. Es decir, buscamos para qué valores de
$\mora{k_{c}}$ existe por lo menos un polo del sistema cuya parte real sea $0$.

Podemos observar del L.G.R. (\autoref{fig:lgr-f}) que existirá un valor positivo
de $\mora{k_{c}}$ para el cual el sistema tendrá dos polos complejos conjugados
con parte real $0$, mientras que para valores negativos de $\mora{k_{c}}$, habrá
un solo valor que hará que el sistema tenga un polo real en el origen.

Entonces, utilizamos el comando \verb|margin| de \textit{MATLAB}, el cual, dado
un sistema en lazo directo, nos entrega el valor de la ganancia para el cual el
sistema es marginalmente estable (véase \autoref{lst:preg-f}). Nótese que para
encontrar el valor negativo de $\mora{k_{c}}$, se multiplicó el sistema por $-1$,
puesto que por defecto, \verb|margin| sólo encuentra valores positivos de ganancia.
Los valores encontrados son entonces,
\begin{equation}
    \boxed{\mora{k_{c}} = 0.0222} \quad \text{y} \quad \boxed{\mora{k_{c}} = -0.0088}
\end{equation}

\begin{figure}[ht]
  \centering
  \input{Diagramas/lgr_f.tex}
  \caption{Lugar geométrico de las raíces para el sistema con controlador proporcional.}
  \label{fig:lgr-f}
\end{figure}

Luego, determinamos los polos del sistema para estas ganancias, y observamos que,
\begin{multicols}{2}
    Polos para $\mora{k_{c}} = 22.2\unit{m}$
    \begin{itemize}
        \item $\lambda_{0} = -2.8345$
        \item $\lambda_{1} = +12.1792j$
        \item $\lambda_{2} = -12.1792j$
    \end{itemize}

    Polos para $\mora{k_{c}} = -8.8\unit{m}$
    \begin{itemize}
        \item $\lambda_{0} = 0$
        \item $\lambda_{1} = -1.4173 + 7.2161j$
        \item $\lambda_{2} = -1.4173 - 7.2161j$
    \end{itemize}
\end{multicols}

\subsection{Comentarios}

De los valores de ganancia obtenidos, podemos comentar que, si bien el sistema
es marginalmente estable para $\mora{k_{c}} = -8.8\unit{m}$, la oscilación de
este no será sostenida, puesto que el polo que hace al sistema marginalmente
estable no tiene componente imaginaria. Entonces, la oscilación del sistema
estará dada únicamente por los polos $\lambda_{1,2}$.

Por otro lado, cuando la ganancia es de $\mora{k_{c}} = 22.2\unit{m}$, el
sistema sí mantendrá una oscilación sostenida, y lo hará a una frecuencia de
$\frac{\mathfrak{Im}(\lambda_{1})}{2\pi} = 1.94\ \unit{Hz}$.
