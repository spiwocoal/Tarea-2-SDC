\section{Pregunta \texttt{f)}}\label{pregunta-f}
\subsection{Desarrollo}

Para este apartado nos piden determinar los $\nara{k_{c}}$ para los cuales el L.G.R. tiene por lo menos una raíz marginalmente estable. 

Nos ayudaremos de \textit{MATLAB}, con
el código disponible en \autoref{lst:problema-F}, para hacer los cálculos
de manera simbólica.

Entonces, empezamos por obtener la función de transferencia del sistema para
cualquier $\nara{k_{c}}$. Para esto, usamos lo visto en \eqref{eq:fdt-lc}.

Símbolicamente entonces se tiene que la función de transferencia es
\begin{equation}
    \frac{G(s)}{1 + G(s)r(s)} = \dfrac{\nara{k_ck_a} \frac{\nara{\nara{b_{0\omega}}}\left(\nara{b_{1\psi}}s + \nara{b_{0\psi}}\right)}
    {\left(\nara{a_{1\omega}}s + \nara{a_{0\omega}}\right)\left(\nara{a_{2\psi}}s^{2} + \nara{a_{1\psi}}s + \nara{a_{0\psi}}\right)}}{1+\nara{k_ck_a} \frac{\nara{\nara{b_{0\omega}}}\left(\nara{b_{1\psi}}s + \nara{b_{0\psi}}\right)}
    {\left(\nara{a_{1\omega}}s + \nara{a_{0\omega}}\right)\left(\nara{a_{2\psi}}s^{2} + \nara{a_{1\psi}}s + \nara{a_{0\psi}}\right)}\nara{k_{st}}}
\end{equation}

Reordenadando la expresión utilizando \textit{MATLAB} obtenemos:
\begin{equation}\label{eq:fdt-lc-1}
    \frac{(\nara{k_a k_a b_{1\psi} b_{0\psi}})s + \nara{k_a k_a b_{0\psi} b_{0\omega}}}
    {(\nara{a_{2\psi} a_{1\omega}})s^3 + (\nara{a_{1\psi} a_{1\omega} + a_{2\psi} a_{0\omega}})s^2 + (\nara{a_{0\psi} a_{1\omega} + a_{1\psi} a_{0\omega} + k_a k_c k_{st} b_{1\psi} b_{0\omega}})s + \nara{a_{0\psi} a_{0\omega} + k_a k_c k_{st} b_{0\psi} b_{0\omega}}}
\end{equation}

Nos enfocamos entonces en el denominador que es donde se ecuentran lo polos y
utilizamos el criterio de Routh-Hurwitz para analizar la estabilidad. Trabajaremos
entonces con el denominador de \eqref{eq:fdt-lc-1}, donde $\nara{a_3},\nara{a_2},\nara{a_1},\nara{a_0}$
representan los coeficientes del polinomio:
\begin{align}
    P(s) = \nara{a_3} s^3 + \nara{a_2} s^2 + \nara{a_1} s + \nara{a_0} &
\end{align}

Construimos entonces la tabla para el criterio de Routh-Hurwitz, y observamos
que no hay ceros presentes en la columan pivote:
\begin{equation}
  \begin{array}{c|cc}
    s^3 & \nara{a_3} & \nara{a_1} \\
    s^2 & \nara{a_2} & \nara{a_0} \\
    s^1 & \nara{b_1} & 0 \\
    s^0 & \nara{b_0} & 
  \end{array}
\end{equation}

Donde sabemos que:
\begin{align}
  \nara{b_1} = \frac{\nara{a_2 a_1} - \nara{a_3 a_0}}{\nara{a_2}},\quad
  \nara{b_0} = \nara{a_0}
\end{align}

Para tener estabilidad en el sistema, los coeficientes de la primera columna
deben ser todos positivos. Entonces, la condición $\nara{b_1} > 0$ garantiza
que el sistema vaya a ser estable.

Reemplazamos los coeficientes y obtenemos
\begin{equation}
    \nara{b_1}=\frac{
        \left(\nara{a_{1\psi} a_{1\omega}} + \nara{a_{2\psi} a_{0\omega}}\right)
        \left(\nara{a_{0\psi} a_{1\omega}} + \nara{a_{1\psi} a_{0\omega}} + \nara{k_a k_c k_{st} b_{1\psi} b_{0\omega}}\right) 
        - \nara{a_{2\psi} a_{1\omega}} 
        \left(\nara{a_{0\psi} a_{0\omega}} + \nara{k_a k_c k_{st} b_{0\psi} b_{0\omega}}\right)
    }{
        \nara{a_{1\psi} a_{1\omega}} + \nara{a_{2\psi} a_{0\omega}}
    } > 0
\end{equation}

Resolvemos entonces para $\nara{k_{c}}$, y vemos que para que el sistema sea
estable se necesita que
\begin{align}
       \nara{k_{c}} &< \dfrac{-(\nara{a_{1\psi} a_{1\omega}}(\nara{a_{0\psi} a_{1\omega}} + \nara{a_{1\psi} a_{0\omega}}) + \nara{ a_{2\psi} a_{0\omega} }(\nara{a_{0\psi} a_{1\omega}} + \nara{a_{1\psi} a_{0\omega}}) - \nara{a_{2\psi} a_{1\omega} a_{0\psi} a_{0\omega}})}  {(\nara{a_{1\psi} a_{1\omega} k_a k_{st} b_{1\psi} b_{0\omega}}) + (\nara{a_{2\psi} a_{0\omega} k_a k_{st} b_{1\psi} b_{0\omega}}) - (\nara{a_{2\psi} a_{1\omega} k_a k_{st} b_{0\psi} b_{0\omega}})} \\
  \iff \nara{k_{c}} &< 0.0222
\end{align}

Para que el sistema sea estable entonces se debe cumplir lo anterior. Pero
como solo nos interesa que sea marginalmente estable, entonces el $\nara{k_{c}}$
que hace que el sistema lo sea es:
\begin{equation}
    \boxed{\nara{k_c} = 0.0222}
\end{equation}


\FloatBarrier
\subsection{Comentarios}


\textbf{!AQUÍ COMENTARIOS NO OLVIDAR!}

Posibles Comentarios
\begin{itemize}
    \item Qué parámetro define la ganancia
    \item Es la ganancia única
    \item Siempre es posible que exista una ganancia
    \item El sistema oscila para esta ganancia
    \item Qué parámetro define la frecuencia de la oscilación
    \item Qué parámetro define la amplitud de la oscilación
    \item Cómo cambian los resultados para otro punto de operación
\end{itemize}