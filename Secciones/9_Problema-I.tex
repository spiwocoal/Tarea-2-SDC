\section{Pregunta \texttt{i)}}\label{pregunta-i}
\subsection{Desarrollo}

\subsubsection{C} %después lo borramos
 Para hacer \hyperref[pregunta-c]{\texttt{c)}} en esta pregunta que nos piden encontrar los los polos y ceros en L.D. y en L.C lo hacemos directamente con \textit{MATLAB} con el código que se ve en \autoref{lst:Cod_problema_Ic} donde obtenemos que:

 \begin{itemize}
  \item \textbf{L.D.} = \( \mora{k_{c}} \nara{ k_{a}} z^{-1} \cdot \frac{\rojo{\psi}(z)}{\verd{v_i}(z)} \nara{k_{st}} \) 
  
  \begin{multicols}{2}
    \textbf{Polos}
    \begin{itemize}
      \item \(P_{0} = 0 \)
      \item \(P_{2} = 0.7386\)
      \item \(P_{3,4} = -0.1745 \pm 0.8588i\) 
    \end{itemize}
    \columnbreak
    \textbf{Cero}
    \begin{itemize}
        \item \(Z_0 = 0.4062\)
        \item \(Z_1 = -1.0472\)
    \end{itemize}
  \end{multicols}

  \item \textbf{L.C} =  \(\frac{  \mora{k_{c}} \nara{ k_{a}} z^{-1} \cdot \frac{\rojo{\psi}(z)}{\verd{v_i}(z)}} {1+ \mora{k_{c}} \nara{ k_{a}} z^{-1} \cdot \frac{\rojo{\psi}(z)}{\verd{v_i}(z)} \nara{k_{st}} }  \)
  
  \begin{multicols}{2}
    \textbf{Polos}
    \begin{itemize}
      \item \(P_{0} = 0.6466 \)
      \item \(P_{2} = -0.1565\)
      \item \(P_{3,4} = -0.0503 \pm 0.9297i\) 
    \end{itemize}
    \columnbreak
    \textbf{Cero}
    \begin{itemize}
        \item \(Z_0 = 0.4062\)
        \item \(Z_1 = -1.0472\)
    \end{itemize}
  \end{multicols}
 \end{itemize}



\FloatBarrier
\subsubsection{D}%después lo borramos

\begin{figure}[ht]
    \centering
    % This file was created by matlab2tikz.
%
%The latest updates can be retrieved from
%  http://www.mathworks.com/matlabcentral/fileexchange/22022-matlab2tikz-matlab2tikz
%where you can also make suggestions and rate matlab2tikz.
%
\begin{tikzpicture}

\begin{axis}[%
width=5.871cm,
height=6cm,
at={(0cm,0cm)},
scale only axis,
separate axis lines,
every outer x axis line/.append style={white!15!black},
every x tick label/.append style={font=\color{white!15!black}},
every x tick/.append style={white!15!black},
xmin=-3,
xmax=2,
every outer y axis line/.append style={white!15!black},
every y tick label/.append style={font=\color{white!15!black}},
every y tick/.append style={white!15!black},
ymin=-2.5,
ymax=2.5,
axis background/.style={fill=white},
legend pos=north west
]
  \draw[dotted, draw=gray] (axis cs:0,0) circle [radius=1];
\addplot [color=gray, dotted, forget plot]
  table[]{Diagramas/data/lgr_i_neg-1.tsv};
\addplot [color=gray, dotted, forget plot]
  table[]{Diagramas/data/lgr_i_neg-2.tsv};

\addplot [color=red, dashed, line width=1.5pt]
  table[]{Diagramas/data/lgr_i_neg-3.tsv};
\addplot [color=cyan, dashed, line width=1.5pt, forget plot]
  table[]{Diagramas/data/lgr_i_neg-4.tsv};
\addplot [color=ForestGreen, dashed, line width=1.5pt, forget plot]
  table[]{Diagramas/data/lgr_i_neg-5.tsv};
\addlegendentry{$\mora{k_{c}} < 0$}


\addplot [color=red, line width=1.5pt]
  table[]{Diagramas/data/lgr_i_pos-3.tsv};
\addplot [color=cyan, line width=1.5pt, forget plot]
  table[]{Diagramas/data/lgr_i_pos-4.tsv};
\addplot [color=ForestGreen, line width=1.5pt, forget plot]
  table[]{Diagramas/data/lgr_i_pos-5.tsv};
\addlegendentry{$\mora{k_{c}} > 0$}

\addplot[only marks, mark=o, thick, mark size=2.0pt, draw=Fuchsia, forget plot] table[]{Diagramas/data/lgr_i_neg-6.tsv};
\addplot[only marks, mark=x, thick, mark size=2.5pt, draw=Fuchsia, forget plot] table[]{Diagramas/data/lgr_i_neg-7.tsv};
\end{axis}

\begin{axis}[%
width=7.334cm,
height=7.334cm,
at={(-0.766cm,-0.784cm)},
scale only axis,
xmin=0,
xmax=1,
ymin=0,
ymax=1,
axis line style={draw=none},
ticks=none,
axis x line*=bottom,
axis y line*=left
]
\end{axis}
\end{tikzpicture}%

    \caption{Lugar geométrico de las raíces para el sistema. Controlador proporcional.}
    \label{fig:lgr-i}
  \end{figure}

\FloatBarrier
\subsubsection{F}%después lo borramos

\FloatBarrier
\subsubsection{G}%después lo borramos

\FloatBarrier
\subsection{Comentarios}


\textbf{!AQUÍ COMENTARIOS NO OLVIDAR!}
Posibles Comentarios
\begin{itemize}
    \item Idem (c), (d), (f) y (g) comparando con el controlador de ganancia tiempo continuo
\end{itemize}
