\section{Pregunta \texttt{i)}}\label{pregunta-i}
\subsection{Desarrollo}

\begin{figure}[h]
  \centering
  \begin{tikzpicture}
    % Suma
    \node[draw,
        circle,
        minimum size=0.6cm,
        fill=white!50
    ] (sum) at (0,0){};
     
    \draw (sum.north east) -- (sum.south west)
        (sum.north west) -- (sum.south east);
     
    \draw (sum.north east) -- (sum.south west)
    (sum.north west) -- (sum.south east);
     
    \node[left=-2pt] at (sum.center){\tiny $+$};
    \node[below=-1pt] at (sum.center){\tiny $-$};
     
    % Controlador
    \node [block,
        color=Emerald,
        fill=white,
        text=black,
        right=0.5cm of sum
      ]  (controller) {$h_{c}(z)$};

    % Retardo
    \node [block,
        color=Emerald,
        fill=white,
        text=black,
        right=1cm of controller
      ]  (delay) {$z^{-1}$};

    % Retentor
    \node [block,
        color=Orange,
        fill=white,
        right=1cm of delay
      ]  (holder) {S/H};
    
    % Actuador
    \node [block,
        color=Blue,
        fill=white, 
        right=1cm of holder
      ] (actuator) {$\nara{k_{a}}$};
     
    % Sistema H(s)
    \node [block,
        color=Blue,
        fill=white, 
        text=black,
        right=1cm of actuator
      ] (system) {$\rojo{\psi}(s) \mathbin{/} \verd{v_{i}}(s)$};

    % Muestreador
    \node [block,
        color=Orange,
        fill=white, 
        below= 1cm of holder
      ]  (sampler) {S};
     
    % Sensor/Transmisor
    \node [block,
        color=Blue,
        fill=white, 
        right= 2.5cm of sampler
      ]  (sensor) {$\nara{k_{st}}$};
     
    % Arrows with text label
    \draw[-stealth, Emerald] (sum.east) -- (controller.west)
        node[midway,above,text=black]{$e$};

    \draw[-stealth, Emerald] (controller.east) -- (delay.west)
        node[midway,above]{$\azul{w_{d}}$};

    \draw[-stealth, Emerald] (delay.east) -- (holder.west)
        node[midway,above]{};

    \draw[-stealth, Blue] (holder.east) -- (actuator.west) 
        node[midway,above]{$\azul{w}$};
    
    \draw[-stealth, Blue] (actuator.east) -- (system.west) 
        node[midway,above]{$\verd{v_{i}}$};
     
    \draw[-stealth, Blue] (system.east) -- ++ (1.25,0) 
        node[midway](output){}node[midway,above]{$\rojo{\psi}$};
     
    \draw[-stealth, Blue] (output.center) |- (sensor.east);
     
    \draw[-stealth, Blue] (sensor.west) -- (sampler.east) 
        node[midway,above]{$\rojo{\psi_{m}}$};

    \draw[-stealth, Emerald] (sampler.west) -| (sum.south) 
        node[near end,left]{$\rojo{\psi_{md}}$};
     
    \draw[Emerald] (sum.west) -- ++(-1,0) 
        node[midway,above]{$\rojo{\psi_{d}}$};
  \end{tikzpicture}
  \caption{Lazo de control para el sistema híbrido.}\label{fig:diag-lc-hibrido}
\end{figure}

\subsubsection{C} %después lo borramos
 Para hacer \hyperref[pregunta-c]{\texttt{c)}} en esta pregunta que nos piden encontrar los los polos y ceros en L.D. y en L.C lo hacemos directamente con \textit{MATLAB} con el código que se ve en \autoref{lst:Cod_problema_Ic} donde obtenemos que:

 \begin{itemize}
  \item \textbf{L.D.} = \( \mora{k_{c}} \nara{ k_{a}} z^{-1} \cdot \frac{\rojo{\psi}(z)}{\verd{v_i}(z)} \nara{k_{st}} \) 
  
  \begin{multicols}{2}
    \textbf{Polos}
    \begin{itemize}
      \item \(P_{0} = 0 \)
      \item \(P_{2} = 0.7386\)
      \item \(P_{3,4} = -0.1745 \pm 0.8588i\) 
    \end{itemize}
    \columnbreak
    \textbf{Cero}
    \begin{itemize}
        \item \(Z_0 = 0.4062\)
        \item \(Z_1 = -1.0472\)
    \end{itemize}
  \end{multicols}

  \item \textbf{L.C} =  \(\frac{  \mora{k_{c}} \nara{ k_{a}} z^{-1} \cdot \frac{\rojo{\psi}(z)}{\verd{v_i}(z)}} {1+ \mora{k_{c}} \nara{ k_{a}} z^{-1} \cdot \frac{\rojo{\psi}(z)}{\verd{v_i}(z)} \nara{k_{st}} }  \)
  
  \begin{multicols}{2}
    \textbf{Polos}
    \begin{itemize}
      \item \(P_{0} = 0.6466 \)
      \item \(P_{2} = -0.1565\)
      \item \(P_{3,4} = -0.0503 \pm 0.9297i\) 
    \end{itemize}
    \columnbreak
    \textbf{Cero}
    \begin{itemize}
        \item \(Z_0 = 0.4062\)
        \item \(Z_1 = -1.0472\)
    \end{itemize}
  \end{multicols}
 \end{itemize}



\FloatBarrier
\subsubsection{D}%después lo borramos

\begin{figure}[ht]
    \centering
    % This file was created by matlab2tikz.
%
%The latest updates can be retrieved from
%  http://www.mathworks.com/matlabcentral/fileexchange/22022-matlab2tikz-matlab2tikz
%where you can also make suggestions and rate matlab2tikz.
%
\begin{tikzpicture}

\begin{axis}[%
width=5.871cm,
height=6cm,
at={(0cm,0cm)},
scale only axis,
separate axis lines,
every outer x axis line/.append style={white!15!black},
every x tick label/.append style={font=\color{white!15!black}},
every x tick/.append style={white!15!black},
xmin=-2,
xmax=2,
every outer y axis line/.append style={white!15!black},
every y tick label/.append style={font=\color{white!15!black}},
every y tick/.append style={white!15!black},
ymin=-2,
ymax=2,
axis background/.style={fill=white},
legend pos=outer north east
]
\draw[dotted, draw=gray] (axis cs:0,0) circle [radius=1];
\addplot [color=gray, dotted, forget plot]
  table[]{Diagramas/data/lgr_i_neg-1.tsv};
\addplot [color=gray, dotted, forget plot]
  table[]{Diagramas/data/lgr_i_neg-2.tsv};

\addplot [dashed, color=cyan, line width=1.5pt, forget plot]
  table[]{Diagramas/data/lgr_i_neg-3.tsv};
\addplot [dashed, color=red, line width=1.5pt]
  table[]{Diagramas/data/lgr_i_neg-4.tsv};
\addplot [dashed, color=ForestGreen, line width=1.5pt, forget plot]
  table[]{Diagramas/data/lgr_i_neg-5.tsv};
\addplot [dashed, color=RoyalBlue, line width=1.5pt, forget plot]
  table[]{Diagramas/data/lgr_i_neg-6.tsv};
\addlegendentry{$\mora{k_{c}} < 0$}

\addplot [color=cyan, line width=1.5pt, forget plot]
  table[]{Diagramas/data/lgr_i_pos-3.tsv};
\addplot [color=RoyalBlue, line width=1.5pt, forget plot]
  table[]{Diagramas/data/lgr_i_pos-4.tsv};
\addplot [color=red, line width=1.5pt]
  table[]{Diagramas/data/lgr_i_pos-5.tsv};
\addplot [color=ForestGreen, line width=1.5pt, forget plot]
  table[]{Diagramas/data/lgr_i_pos-6.tsv};
\addlegendentry{$\mora{k_{c}} > 0$}

\addplot[only marks, mark=o, mark options={}, mark size=2.0pt, thick, draw=Fuchsia, forget plot] table[]{Diagramas/data/lgr_i_neg-7.tsv};
\addplot[only marks, mark=x, mark options={}, mark size=2.5pt, thick, draw=Fuchsia, forget plot] table[]{Diagramas/data/lgr_i_neg-8.tsv};
\end{axis}
\end{tikzpicture}%

    \caption{Lugar geométrico de las raíces para el sistema. Controlador proporcional.}
    \label{fig:lgr-i}
  \end{figure}

\FloatBarrier
\subsubsection{F}%después lo borramos
Ahora nos piden calcular cuando por lo menos se una raíz marginalmente estable. 
Donde lo realizamos en \textit{MATLAB}, con el comando \verb|allmargin|  y  \verb|.GainMargin|, que nos entrega directamente las ganancias que requerimos, esto se ve en el \autoref{lst:Cod_problema_If}. Y ademas sus raíces correspondientes en L.C las calculamos con el mismo código en \autoref{lst:Cod_problema_Ic} solo cambiando el \(k_c\) correspondiente y tomando en cuenta el L.C tal que los \( \mora{k_c}\) y sus raíces serian los  serian los siguientes : 

\begin{multicols}{2}
  \begin{itemize}
      \item \(\mora{k_{c1}} = 0.0069\) 

        \textbf{Polos}
        \begin{align}
        \lambda_{0,1} &=0.0221 \pm 1.0006i \\
        \lambda_{2} &= -0.2526 \\
        \lambda_{3} &= 0.5979 \\
        \end{align}

      \item \(\mora{k_{c2}} = 0.7214\)

      \textbf{Polos}
      \begin{align}
        \lambda_{0,1} &= 0.4903 \pm 6.2005i \\
        \lambda_{2} &= -1.0000 \\
        \lambda_{3} &= 0.4089 \\
        \end{align}
  \end{itemize}
  \columnbreak
  \begin{itemize}
      \item \(\mora{k_{c3}} = -0.0088\)

      \textbf{Polos}
      \begin{align}
        \lambda_{0,1} &= -0.4178 \pm 0.8245i \\
        \lambda_{2} &= 0.9990 \\
        \lambda_{3} &= 0.2260 \\
        \end{align}

      \item \(\mora{k_{c4}} = -0.0164\)

      \textbf{Polos}
      \begin{align}
        \lambda_{0,1} &= -0.5609 \pm 0.8283i \\
        \lambda_{2} &= 1.2156  \\
        \lambda_{3} &= 0.2956  \\
        \end{align}
  \end{itemize}
\end{multicols}


Los ceros del LC independiente del \(\mora{k_c}\) resultaban ser $Z_{0} = 0.4062 $ y $Z_{1} = -1.0472$ 



\FloatBarrier
\subsubsection{G}%después lo borramos

Ahora bien analizando los \(\mora{k_c}\) obtenidos y su L.G.R llegamos a la
conclusion de que el rango en donde el sistema es estable es:
\begin{equation}
  \boxed{ -8.8\unit{m} \leq \mora{k_{c}} \leq 6.9\unit{m} }
\end{equation}

\FloatBarrier
\subsection{Comentarios}


\textbf{!AQUÍ COMENTARIOS NO OLVIDAR!}
Posibles Comentarios
\begin{itemize}
    \item Idem (c), (d), (f) y (g) comparando con el controlador de ganancia tiempo continuo
\end{itemize}
