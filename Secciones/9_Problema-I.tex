\section{Pregunta \texttt{i)}}\label{pregunta-i}
\subsection{Desarrollo}

A continuación, se nos pide considerar el sistema híbrido que se puede observar
en la \autoref{fig:diag-lc-hibrido}. En primer lugar, nos interesa estudiar el
comportamiento del sistema cuando el controlador es $h_{c}(z) = \mora{k_{c}}$.

\begin{figure}[h]
  \centering
  \begin{tikzpicture}
    % Suma
    \node[draw,
        circle,
        minimum size=0.6cm,
        fill=white!50
    ] (sum) at (0,0){};
     
    \draw (sum.north east) -- (sum.south west)
        (sum.north west) -- (sum.south east);
     
    \draw (sum.north east) -- (sum.south west)
    (sum.north west) -- (sum.south east);
     
    \node[left=-2pt] at (sum.center){\tiny $+$};
    \node[below=-1pt] at (sum.center){\tiny $-$};
     
    % Controlador
    \node [block,
        color=Emerald,
        fill=white,
        text=black,
        right=0.5cm of sum
      ]  (controller) {$h_{c}(z)$};

    % Retardo
    \node [block,
        color=Emerald,
        fill=white,
        text=black,
        right=1cm of controller
      ]  (delay) {$z^{-1}$};

    % Retentor
    \node [block,
        color=Orange,
        fill=white,
        right=1cm of delay
      ]  (holder) {S/H};
    
    % Actuador
    \node [block,
        color=Blue,
        fill=white, 
        right=1cm of holder
      ] (actuator) {$\nara{k_{a}}$};
     
    % Sistema H(s)
    \node [block,
        color=Blue,
        fill=white, 
        text=black,
        right=1cm of actuator
      ] (system) {$\rojo{\psi}(s) \mathbin{/} \verd{v_{i}}(s)$};

    % Muestreador
    \node [block,
        color=Orange,
        fill=white, 
        below= 1cm of holder
      ]  (sampler) {S};
     
    % Sensor/Transmisor
    \node [block,
        color=Blue,
        fill=white, 
        right= 2.5cm of sampler
      ]  (sensor) {$\nara{k_{st}}$};
     
    % Arrows with text label
    \draw[-stealth, Emerald] (sum.east) -- (controller.west)
        node[midway,above,text=black]{$e$};

    \draw[-stealth, Emerald] (controller.east) -- (delay.west)
        node[midway,above]{$\azul{w_{d}}$};

    \draw[-stealth, Emerald] (delay.east) -- (holder.west)
        node[midway,above]{};

    \draw[-stealth, Blue] (holder.east) -- (actuator.west) 
        node[midway,above]{$\azul{w}$};
    
    \draw[-stealth, Blue] (actuator.east) -- (system.west) 
        node[midway,above]{$\verd{v_{i}}$};
     
    \draw[-stealth, Blue] (system.east) -- ++ (1.25,0) 
        node[midway](output){}node[midway,above]{$\rojo{\psi}$};
     
    \draw[-stealth, Blue] (output.center) |- (sensor.east);
     
    \draw[-stealth, Blue] (sensor.west) -- (sampler.east) 
        node[midway,above]{$\rojo{\psi_{m}}$};

    \draw[-stealth, Emerald] (sampler.west) -| (sum.south) 
        node[near end,left]{$\rojo{\psi_{md}}$};
     
    \draw[Emerald] (sum.west) -- ++(-1,0) 
        node[midway,above]{$\rojo{\psi_{d}}$};
  \end{tikzpicture}
  \caption{Lazo de control para el sistema híbrido.}\label{fig:diag-lc-hibrido}
\end{figure}

Entonces, empezamos por determinar el valor de los polos y ceros del sistema en
L.D. y en L.C. Para esto, al igual que como se realizó anteriormente, nos apoyamos
directamente de las capacidades de \textit{MATLAB} para trabajar con sistemas
(véase \autoref{lst:Cod_problema_Hc}). Entonces, los valores obtenidos son:

\begin{itemize}
  \item \textbf{L.D.} = \( \mora{k_{c}} \nara{ k_{a}} z^{-1} \cdot \frac{\rojo{\psi}(z)}{\verd{v_i}(z)} \nara{k_{st}} \) 
 
  \begin{multicols}{2}
    \textbf{Polos}
    \begin{itemize}
      \item \(P_{0} = 0 \)
      \item \(P_{1} = 0.7386\)
      \item \(P_{2,3} = -0.1745 \pm 0.8588i\) 
    \end{itemize}
    \columnbreak
    \textbf{Cero}
    \begin{itemize}
      \item \(Z_0 = 0.4062\)
      \item \(Z_1 = -1.0472\)
    \end{itemize}
  \end{multicols}
  
  \item \textbf{L.C} =  \(\frac{  \mora{k_{c}} \nara{ k_{a}} z^{-1} \cdot \frac{\rojo{\psi}(z)}{\verd{v_i}(z)}} {1+ \mora{k_{c}} \nara{ k_{a}} z^{-1} \cdot \frac{\rojo{\psi}(z)}{\verd{v_i}(z)} \nara{k_{st}} }  \)
  
  \begin{multicols}{2}
    \textbf{Polos}
    \begin{itemize}
      \item \(P_{0} = 0.6466 \)
      \item \(P_{1} = -0.1565\)
      \item \(P_{2,3} = -0.0503 \pm 0.9297i\) 
    \end{itemize}
    \columnbreak
    \textbf{Cero}
    \begin{itemize}
      \item \(Z_0 = 0.4062\)
      \item \(Z_1 = -1.0472\)
    \end{itemize}
  \end{multicols}
\end{itemize}

\begin{figure}[ht]
    \centering
    % This file was created by matlab2tikz.
%
%The latest updates can be retrieved from
%  http://www.mathworks.com/matlabcentral/fileexchange/22022-matlab2tikz-matlab2tikz
%where you can also make suggestions and rate matlab2tikz.
%
\begin{tikzpicture}

\begin{axis}[%
width=5.871cm,
height=6cm,
at={(0cm,0cm)},
scale only axis,
separate axis lines,
every outer x axis line/.append style={white!15!black},
every x tick label/.append style={font=\color{white!15!black}},
every x tick/.append style={white!15!black},
xmin=-2,
xmax=2,
every outer y axis line/.append style={white!15!black},
every y tick label/.append style={font=\color{white!15!black}},
every y tick/.append style={white!15!black},
ymin=-2,
ymax=2,
xlabel={$\mathfrak{Re}$},
ylabel={$\mathfrak{Im}$},
axis background/.style={fill=white},
legend cell align=left,
legend pos=outer north east
]
\draw[dotted, draw=gray] (axis cs:0,0) circle [radius=1];
\addplot [color=gray, dotted, forget plot]
  table[]{Diagramas/data/lgr_i_neg-1.tsv};
\addplot [color=gray, dotted, forget plot]
  table[]{Diagramas/data/lgr_i_neg-2.tsv};

\addplot [dashed, color=cyan, line width=1.5pt, forget plot]
  table[]{Diagramas/data/lgr_i_neg-3.tsv};
\addplot [dashed, color=red, line width=1.5pt]
  table[]{Diagramas/data/lgr_i_neg-4.tsv};
\addplot [dashed, color=ForestGreen, line width=1.5pt, forget plot]
  table[]{Diagramas/data/lgr_i_neg-5.tsv};
\addplot [dashed, color=RoyalBlue, line width=1.5pt, forget plot]
  table[]{Diagramas/data/lgr_i_neg-6.tsv};
\addlegendentry{$\mora{k_{c}} < 0$}

\addplot [color=cyan, line width=1.5pt, forget plot]
  table[]{Diagramas/data/lgr_i_pos-3.tsv};
\addplot [color=RoyalBlue, line width=1.5pt, forget plot]
  table[]{Diagramas/data/lgr_i_pos-4.tsv};
\addplot [color=red, line width=1.5pt]
  table[]{Diagramas/data/lgr_i_pos-5.tsv};
\addplot [color=ForestGreen, line width=1.5pt, forget plot]
  table[]{Diagramas/data/lgr_i_pos-6.tsv};
\addlegendentry{$\mora{k_{c}} > 0$}

\addplot[only marks, mark=o, mark options={}, mark size=2.0pt, thick, draw=Fuchsia, forget plot] table[]{Diagramas/data/lgr_i_neg-7.tsv};
\addplot[only marks, mark=x, mark options={}, mark size=2.5pt, thick, draw=Fuchsia, forget plot] table[]{Diagramas/data/lgr_i_neg-8.tsv};

\addplot[only marks, mark=square, mark size=2.0pt, thick, draw=Gray] table[row sep=crcr]{%
  0 0\\
  0.7386 0\\
  -0.1745 -0.8588\\
  -0.1745 0.8588\\
  };
\addlegendentry{Polos en L.D. para $\mora{k_{c}}=4\unit{m}$}

\addplot[only marks, mark=square, mark size=2.0pt, thick, draw=Lavender] table[row sep=crcr]{%
  0.6466 0\\
  -0.1565 0\\
  -0.0503 -0.9297\\
  -0.0503 0.9297\\
  };
\addlegendentry{Polos en L.C. para $\mora{k_{c}}=4\unit{m}$}
\end{axis}
\end{tikzpicture}%

    \caption{Lugar geométrico de las raíces para el sistema. Controlador proporcional.}
    \label{fig:lgr-i1}
\end{figure}

Luego, se dibuja el lugar geométrico de las raíces, tal como se lleva haciendo
hasta el momento, con el comando \verb|rlocus|. (\autoref{fig:lgr-i1})
\newpage

Luego, nos interesa determinar los valores de ganancia para los cuales el sistema
es marginalmente estable. Se puede observar de la \autoref{fig:lgr-i1} que habrán
$4$ valores de ganancia para los cuales esto ocurre. Entonces, utilizamos el comando
de \textit{MATLAB} \verb|allmargin| el cual nos retornará todas las ganancias
para un sistema marginalmente estable (\autoref{lst:Cod_problema_If}). Además,
calculamos las raíces en L.C. para cada uno de las ganancias obtenidas.

\begin{multicols}{2}
  \begin{itemize}
  \item Polos para \(\mora{k_{c1}} = 0.0069\)
    \begin{itemize}
      \item $\lambda_{0,1} =0.0221 \pm 1.0006j$
      \item $\lambda_{2} = -0.2526$
      \item $\lambda_{3} = 0.5$
    \end{itemize}

  \item Polos para \(\mora{k_{c2}} = 0.7214\)
    \begin{itemize}
      \item $\lambda_{0,1} = 0.4903 \pm 6.2005j$
      \item $\lambda_{2} = -1$
      \item $\lambda_{3} = 0.4$
    \end{itemize}

  \columnbreak

  \item Polos para \(\mora{k_{c3}} = -0.0088\)
    \begin{itemize}
      \item $\lambda_{0,1} = -0.4178 \pm 0.8245j$
      \item $\lambda_{2} = 0.9990$
      \item $\lambda_{3} = 0.2$
    \end{itemize}

  \item Polos para \(\mora{k_{c4}} = -0.0164\)
    \begin{itemize}
      \item $\lambda_{0,1} = -0.5609 \pm 0.8283j$
      \item $\lambda_{2} = 1.2156$
      \item $\lambda_{3} = 0.2$
    \end{itemize}
  \end{itemize}
\end{multicols}

\begin{figure}[ht]
    \centering
    % This file was created by matlab2tikz.
%
%The latest updates can be retrieved from
%  http://www.mathworks.com/matlabcentral/fileexchange/22022-matlab2tikz-matlab2tikz
%where you can also make suggestions and rate matlab2tikz.
%
\begin{tikzpicture}

\begin{axis}[%
width=5.871cm,
height=6cm,
at={(0cm,0cm)},
scale only axis,
separate axis lines,
every outer x axis line/.append style={white!15!black},
every x tick label/.append style={font=\color{white!15!black}},
every x tick/.append style={white!15!black},
xmin=-2,
xmax=2,
every outer y axis line/.append style={white!15!black},
every y tick label/.append style={font=\color{white!15!black}},
every y tick/.append style={white!15!black},
ymin=-2,
ymax=2,
axis background/.style={fill=white},
legend cell align=left,
legend pos=outer north east
]
\draw[dotted, draw=gray] (axis cs:0,0) circle [radius=1];
\addplot [color=gray, dotted, forget plot]
  table[]{Diagramas/data/lgr_i_neg-1.tsv};
\addplot [color=gray, dotted, forget plot]
  table[]{Diagramas/data/lgr_i_neg-2.tsv};

\addplot [dashed, color=cyan, line width=1.5pt, forget plot]
  table[]{Diagramas/data/lgr_i_neg-3.tsv};
\addplot [dashed, color=red, line width=1.5pt]
  table[]{Diagramas/data/lgr_i_neg-4.tsv};
\addplot [dashed, color=ForestGreen, line width=1.5pt, forget plot]
  table[]{Diagramas/data/lgr_i_neg-5.tsv};
\addplot [dashed, color=RoyalBlue, line width=1.5pt, forget plot]
  table[]{Diagramas/data/lgr_i_neg-6.tsv};
\addlegendentry{$\mora{k_{c}} < 0$}

\addplot [color=cyan, line width=1.5pt, forget plot]
  table[]{Diagramas/data/lgr_i_pos-3.tsv};
\addplot [color=RoyalBlue, line width=1.5pt, forget plot]
  table[]{Diagramas/data/lgr_i_pos-4.tsv};
\addplot [color=red, line width=1.5pt]
  table[]{Diagramas/data/lgr_i_pos-5.tsv};
\addplot [color=ForestGreen, line width=1.5pt, forget plot]
  table[]{Diagramas/data/lgr_i_pos-6.tsv};
\addlegendentry{$\mora{k_{c}} > 0$}

\addplot[only marks, mark=o, mark options={}, mark size=2.0pt, thick, draw=Fuchsia, forget plot] table[]{Diagramas/data/lgr_i_neg-7.tsv};
\addplot[only marks, mark=x, mark options={}, mark size=2.5pt, thick, draw=Fuchsia, forget plot] table[]{Diagramas/data/lgr_i_neg-8.tsv};

\addplot[only marks, mark=square, mark size=2.0pt, thick, draw=Gray] table[row sep=crcr]{%
  0.0221 -1.0006\\
  0.0221 1.0006\\
  -0.2526 0\\
  0.5979 0\\
  };
\addlegendentry{Polos en L.C. para $\mora{k_{c}}=6.9\unit{m}$}

\addplot[only marks, mark=square, mark size=2.0pt, thick, draw=Lavender] table[row sep=crcr]{%
  0.4903 -6.2005\\
  0.4903 6.2005\\
  -1 0\\
  0.4089 0\\
  };
\addlegendentry{Polos en L.C. para $\mora{k_{c}}=721.4\unit{m}$}

\addplot[only marks, mark=square, mark size=2.0pt, thick, draw=Emerald] table[row sep=crcr]{%
  -0.4178 -0.8245\\
  -0.4178 0.8245\\
  1 0\\
  0.2260 0\\
  };
\addlegendentry{Polos en L.C. para $\mora{k_{c}}=-8.8\unit{m}$}

\addplot[only marks, mark=square, mark size=2.0pt, thick, draw=Dandelion] table[row sep=crcr]{%
  -0.5609 -0.8283\\
  -0.5609 0.8283\\
  1.2156 0\\
  0.2956 0\\
  };
\addlegendentry{Polos en L.C. para $\mora{k_{c}}=-16.4\unit{m}$}
\end{axis}
\end{tikzpicture}%

    \caption{Lugar geométrico de las raíces para el sistema. Controlador proporcional.}
    \label{fig:lgr-i2}
\end{figure}

Observamos que los ceros en L.C, independiente del valor dado a \(\mora{k_c}\),
son $Z_{0} = 0.4062 $ y $Z_{1} = -1.0472$. Y luego, ubicamos los polos encontrados
en el lugar geométricos de las raíces (\autoref{fig:lgr-i2}).

Analizando los valores de \(\mora{k_c}\) obtenidos y su L.G.R, concluimos que el
rango en el cual el sistema es estable es aquel en el que los polos permanecen
dentro del círculo unitario. Por lo tanto, el intervalo de estabilidad para \(\mora{k_c}\) es:
\begin{equation}
  \boxed{ -8.8\unit{m} \leq \mora{k_{c}} \leq 6.9\unit{m} }
\end{equation}

\subsection{Comentarios}

Cuando discretizamos el sistema, se observa un cambio significativo en la
dinámica de los polos debido a la discretización del sistema. Donde además
vemos que en comparación con el controlador proporcional en tiempo continuo,
aquí se añade 1 cero más al sistema. Aún así, se observae nuevamente mismo
comportamiento que ya se vió, es decir, que los ceros se mantienen tanto para
L.D. como L.C, y se mantendrán también para los ceros del controlador sumador
en tiempo discreto. La razón de esto es que, al igual que en el tiempo
continuo el comportamiento de los ceros en este caso depende solamente de la
planta. No así como es en el caso de los polos, que vemos que cambian del L.D.
al L.C. puesto que su valor depende de la ganancia del controlador.

También vemos que existen más valores de \( \mora{k_c}\) que hacen que haya por
lo menos un polo marginalmente estable, en comparación con el caso continuo.
Sin embargo, esto no implica que todos los \( \mora{k_c}\) que se obtuvieron
hagan estable al sistema, ya que con algunos de estos, como se puede apreciar
en la \autoref{fig:lgr-i2}, uno o más de los polos queda fuera del el circulo
unitario, e incluso algunos no se llegan a apreciar, ya que se ubicarían fuera
de los límites de los gráficos. Esto entonces nos indica que ese valor de \( \mora{k_c}\) hace del inestable. 

En resumen, estudiando el comportamiento por medio del L.G.R., podemos
visualizar que el sistema al menos posee un valor de \( \mora{k_c} > 0 \) y un
valor de \( \mora{k_c} < 0 \) que hacen que el sistema se comporte de manera
estable, fijando como los márgenes de estabilidad la circunferencia unitaria
debido a que es un sistema discreto. Donde en este caso el rango de
\(\mora{k_c}\) que realiza esto es y asegura el sistema sea estable en el
dominio discreto es \(-8.8\unit{m} \leq \mora{k_{c}} \leq 6.9\unit{m} \)        
