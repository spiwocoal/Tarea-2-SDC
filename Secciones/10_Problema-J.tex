\section{Pregunta \texttt{j)}}\label{pregunta-j}
\subsection{Desarrollo}

Ahora, queremos estudiar el sistema con controlador $h_{c}(z) = \mora{k_{c}} \mathbin{/} (z-1)$.
Entonces, al igual que para la pregunta anterior, empezamos por determinar los
polos y ceros en L.D. y L.C. (véase \autoref{lst:Cod_problema_Jc}). Se observa
que,

\begin{itemize}
  \item \textbf{L.D.} = \( \frac{\mora{k_{c}}}{z-1} \nara{ k_{a}} z^{-1} \cdot \frac{\rojo{\psi}(z)}{\verd{v_i}(z)} \nara{k_{st}} \) 
  
  \begin{multicols}{2}
    \textbf{Polos}
    \begin{itemize}
      \item \(P_{0} = 0 \)
      \item \(P_{1} = 1 \)
      \item \(P_{2} = 0.7386\)
      \item \(P_{3,4} = -0.1745 \pm 0.8588j\) 
    \end{itemize}
    \columnbreak
    \textbf{Cero}
    \begin{itemize}
        \item \(Z_0 = 0.4062\)
        \item \(Z_1 = -1.0472\)
    \end{itemize}
  \end{multicols}
  
  
  \item \textbf{L.C} =  \(\frac{ \frac{\mora{k_{c}}}{z-1} \nara{ k_{a}} z^{-1} \cdot \frac{\rojo{\psi}(z)}{\verd{v_i}(z)}} {1+ \frac{\mora{k_{c}}}{z-1} \nara{ k_{a}} z^{-1} \cdot \frac{\rojo{\psi}(z)}{\verd{v_i}(z)} \nara{k_{st}} }  \)
  
  \begin{multicols}{2}
    \textbf{Polos}
    \begin{itemize}
      \item \(P_{0} = 0.1500 \)
      \item \(P_{1,2} = 0.8598 \pm 0.3415j\)
      \item \(P_{3,4} = -0.2400 \pm 0.7910j\)
    \end{itemize}
    \columnbreak
    \textbf{Cero}
    \begin{itemize}
        \item \(Z_0 = 0.4062\)
        \item \(Z_1 = -1.0472\)
    \end{itemize}
  \end{multicols}
\end{itemize}

Luego, dibujamos su lugar geométrico de las raíces, tal como se observa en la
\autoref{fig:lgr-j1}, y además ubicamos los polos del sistema para $\mora{k_{c}} = 4\unit{m}$.

\begin{figure}[ht]
  \centering
  \input{Diagramas/lgr_j1.tex}
  \caption{Lugar geométrico de las raíces para el sistema. Controlador proporcional.}
  \label{fig:lgr-j1}
\end{figure}

Seguimos por determinar los valores de $\mora{k_{c}}$ para los cuales el sistema
es marginalmente estable y, al igual que para el apartado anterior, se utilizó
el comando \verb|allmargin|. El código es similar al de el \autoref{lst:Cod_problema_If},
pero se le agregó la F. de T. del controlador sumador $1 \mathbin{/} (z-1)$.

Los valores de $\mora{k_{c}}$ encontrados, junto a sus polos correspondientes son:

\begin{multicols}{2}
  \begin{itemize}
  \item Polos para \(\mora{k_{c1}} = 0.0069  \) 
    \begin{itemize}
      \item $\lambda_{0,1} = 0.8746 \pm 0.4826j$
      \item $\lambda_{2,3} = -0.2938 \pm 0.7608j$
      \item $\lambda_{4} = 0.2279$
    \end{itemize}

  \item Polos para \(\mora{k_{c2}} = 0.0519\)
    \begin{itemize}
      \item $\lambda_{0,1} = 1.2145 \pm 1.2225j$
      \item $\lambda_{2,3} = -0.7113 \pm 0.7030j$
      \item $\lambda_{4} = 0.3832$
    \end{itemize}

  \item Polos para \(\mora{k_{c3}} = 0\)
    \begin{itemize}
      \item $\lambda_{0,1} = -0.1745 \pm 0.8588j$
      \item $\lambda_{2} = 0.7386$
      \item $\lambda_{3} = 0$
      \item $\lambda_{4} = 1$
    \end{itemize}

    \columnbreak

  \item Polos para \(\mora{k_{c4}} = -0.0066\)
    \begin{itemize}
      \item $\lambda_{0,1} = -0.1258 \pm 0.9920j$
      \item $\lambda_{2} = 1.3217$
      \item $\lambda_{3} = 0.5272$
      \item $\lambda_{4} = -0.2077$
    \end{itemize}

  \item Polos para \(\mora{k_{c5}} = -1.4428\)
    \begin{itemize}
      \item $\lambda_{0,1} = -1.4173 \pm 3.7587j$
      \item $\lambda_{2} = 4.8171$
      \item $\lambda_{3} = -1$
      \item $\lambda_{4} = 0.4070$
    \end{itemize}
  \end{itemize}
\end{multicols}

Observamos que los ceros en L.C, independiente del valor dado a \(\mora{k_c}\),
son $Z_{0} = 0.4062 $ y $Z_{1} = -1.0472$. Y luego, ubicamos los polos encontrados
en el lugar geométricos de las raíces (\autoref{fig:lgr-j2}).

\begin{figure}[ht]
  \centering
  % This file was created by matlab2tikz.
%
%The latest updates can be retrieved from
%  http://www.mathworks.com/matlabcentral/fileexchange/22022-matlab2tikz-matlab2tikz
%where you can also make suggestions and rate matlab2tikz.
%
\begin{tikzpicture}

\begin{axis}[%
width=5.871cm,
height=6cm,
at={(0cm,0cm)},
scale only axis,
separate axis lines,
every outer x axis line/.append style={white!15!black},
every x tick label/.append style={font=\color{white!15!black}},
every x tick/.append style={white!15!black},
xmin=-2,
xmax=2,
every outer y axis line/.append style={white!15!black},
every y tick label/.append style={font=\color{white!15!black}},
every y tick/.append style={white!15!black},
ymin=-2,
ymax=2,
xlabel={$\mathfrak{Re}$},
ylabel={$\mathfrak{Im}$},
axis background/.style={fill=white},
legend cell align=left,
legend pos=outer north east
]
\draw[dotted, draw=gray] (axis cs:0,0) circle [radius=1];
\addplot [color=gray, dotted, forget plot]
  table[]{Diagramas/data/lgr_j_neg-1.tsv};
\addplot [color=gray, dotted, forget plot]
  table[]{Diagramas/data/lgr_j_neg-2.tsv};

\addplot [dashed, color=cyan, line width=1.5pt, forget plot]
  table[]{Diagramas/data/lgr_j_neg-3.tsv};
\addplot [dashed, color=Peach, line width=1.5pt, forget plot]
  table[]{Diagramas/data/lgr_j_neg-4.tsv};
\addplot [dashed, color=ForestGreen, line width=1.5pt, forget plot]
  table[]{Diagramas/data/lgr_j_neg-5.tsv};
\addplot [dashed, color=RoyalBlue, line width=1.5pt, forget plot]
  table[]{Diagramas/data/lgr_j_neg-6.tsv};
\addplot [dashed, color=red, line width=1.5pt]
  table[]{Diagramas/data/lgr_j_neg-7.tsv};
\addlegendentry{$\mora{k_{c}} < 0$}

\addplot [color=Peach, line width=1.5pt, forget plot]
  table[]{Diagramas/data/lgr_j_pos-3.tsv};
\addplot [color=cyan, line width=1.5pt, forget plot]
  table[]{Diagramas/data/lgr_j_pos-4.tsv};
\addplot [color=red, line width=1.5pt]
  table[]{Diagramas/data/lgr_j_pos-5.tsv};
\addplot [color=ForestGreen, line width=1.5pt, forget plot]
  table[]{Diagramas/data/lgr_j_pos-6.tsv};
\addplot [color=RoyalBlue, line width=1.5pt, forget plot]
  table[]{Diagramas/data/lgr_j_pos-7.tsv};
\addlegendentry{$\mora{k_{c}} > 0$}

\addplot[only marks, mark=o, mark options={}, mark size=2.0pt, draw=Fuchsia, thick, forget plot] table[]{Diagramas/data/lgr_j_neg-8.tsv};
\addplot[only marks, mark=x, mark options={}, mark size=2.5pt, draw=Fuchsia, thick, forget plot] table[]{Diagramas/data/lgr_j_neg-9.tsv};

\addplot[only marks, mark=square, mark size=2.0pt, thick, draw=Gray] table[row sep=crcr]{%
  0.8745 -0.4826\\
  0.8745 0.4826\\
  -0.2938 -0.7608\\
  -0.2938 0.7608\\
  0.2279 0\\
  };
\addlegendentry{Polos en L.C. para $\mora{k_{c}}=6.9\unit{m}$}

\addplot[only marks, mark=square, mark size=2.0pt, thick, draw=Lavender] table[row sep=crcr]{%
  1.2145 -1.2225\\
  1.2145 1.2225\\
  -0.7113 -0.7030\\
  -0.7113 0.7030\\
  0.3832 0\\
  };
\addlegendentry{Polos en L.C. para $\mora{k_{c}}=51.9\unit{m}$}

\addplot[only marks, mark=square, mark size=2.0pt, thick, draw=Emerald] table[row sep=crcr]{%
  -0.1258 -0.992\\
  -0.1258 0.992\\
  1.3217 0\\
  0.5272 0\\
  -0.2077 0\\
  };
\addlegendentry{Polos en L.C. para $\mora{k_{c}}=-6.6\unit{m}$}

\addplot[only marks, mark=square, mark size=2.0pt, thick, draw=Dandelion] table[row sep=crcr]{%
  -1.4173 -3.7587\\
  -1.4173 3.7587\\
  4.8171 0\\
  -1 0\\
  0.4070 0\\
  };
\addlegendentry{Polos en L.C. para $\mora{k_{c}}=-1.4428$}
\end{axis}
\end{tikzpicture}%

  \caption{Lugar geométrico de las raíces para el sistema. Controlador proporcional.}
  \label{fig:lgr-j2}
\end{figure}


Analizando los valores de \(\mora{k_c}\) obtenidos y su L.G.R, concluimos que el
rango de ganancias en el cual el sistema es estable es:
\begin{equation}
  \boxed{ 0 \leq \mora{k_{c}} \leq 6.9\unit{m} }
\end{equation}

\subsection{Comentarios}


\textbf{!AQUÍ COMENTARIOS NO OLVIDAR!}

Posibles Comentarios
\begin{itemize}
    \item Idem (c), (d), (f) y (g) comparando con el controlador con integrador tiempo continuo
\end{itemize}
