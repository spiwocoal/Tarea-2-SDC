\section{Pregunta \texttt{j)}}\label{pregunta-j}
\subsection{Desarrollo}

\subsubsection{C} %después lo borramos

Nuevamente realizamos lo mismo que en  \hyperref[pregunta-i]{\texttt{i)}} para encontrar los polos y ceros en L.D. y en L.C lo hacemos directamente con \textit{MATLAB} con el código que se ve en \autoref{lst:Cod_problema_Jc} donde nos entrega que:

 \begin{itemize}
  \item \textbf{L.D.} = \( \frac{\mora{k_{c}}}{z-1} \nara{ k_{a}} z^{-1} \cdot \frac{\rojo{\psi}(z)}{\verd{v_i}(z)} \nara{k_{st}} \) 
  
  \begin{multicols}{2}
    \textbf{Polos}
    \begin{itemize}
      \item \(P_{0} = 0 \)
      \item \(P_{1} = 1 \)
      \item \(P_{2} = 0.7386\)
      \item \(P_{3,4} = -0.1745 \pm 0.8588i\) 
    \end{itemize}
    \columnbreak
    \textbf{Cero}
    \begin{itemize}
        \item \(Z_0 = 0.4062\)
        \item \(Z_1 = -1.0472\)
    \end{itemize}
  \end{multicols}


  \item \textbf{L.C} =  \(\frac{ \frac{\mora{k_{c}}}{z-1} \nara{ k_{a}} z^{-1} \cdot \frac{\rojo{\psi}(z)}{\verd{v_i}(z)}} {1+ \frac{\mora{k_{c}}}{z-1} \nara{ k_{a}} z^{-1} \cdot \frac{\rojo{\psi}(z)}{\verd{v_i}(z)} \nara{k_{st}} }  \)
  
  \begin{multicols}{2}
    \textbf{Polos}
    \begin{itemize}
      \item \(P_{0} = 0.1500 \)
      \item \(P_{1,2} = 0.8598 \pm 0.3415i\)
      \item \(P_{3,4} = -0.2400 \pm 0.7910i\) 
    \end{itemize}
    \columnbreak
    \textbf{Cero}
    \begin{itemize}
        \item \(Z_0 = 0.4062\)
        \item \(Z_1 = -1.0472\)
    \end{itemize}
  \end{multicols}
 \end{itemize}





\FloatBarrier
\subsubsection{D}%después lo borramos

\begin{figure}[ht]
  \centering
  % This file was created by matlab2tikz.
%
%The latest updates can be retrieved from
%  http://www.mathworks.com/matlabcentral/fileexchange/22022-matlab2tikz-matlab2tikz
%where you can also make suggestions and rate matlab2tikz.
%
\begin{tikzpicture}

\begin{axis}[%
width=5.871cm,
height=6cm,
at={(0cm,0cm)},
scale only axis,
separate axis lines,
every outer x axis line/.append style={white!15!black},
every x tick label/.append style={font=\color{white!15!black}},
every x tick/.append style={white!15!black},
xmin=-2,
xmax=2,
every outer y axis line/.append style={white!15!black},
every y tick label/.append style={font=\color{white!15!black}},
every y tick/.append style={white!15!black},
ymin=-2,
ymax=2,
xlabel={$\mathfrak{Re}$},
ylabel={$\mathfrak{Im}$},
axis background/.style={fill=white},
legend cell align=left,
legend pos=outer north east
]
\draw[dotted, draw=gray] (axis cs:0,0) circle [radius=1];
\addplot [color=gray, dotted, forget plot]
  table[]{Diagramas/data/lgr_j_neg-1.tsv};
\addplot [color=gray, dotted, forget plot]
  table[]{Diagramas/data/lgr_j_neg-2.tsv};

\addplot [dashed, color=cyan, line width=1.5pt, forget plot]
  table[]{Diagramas/data/lgr_j_neg-3.tsv};
\addplot [dashed, color=Peach, line width=1.5pt, forget plot]
  table[]{Diagramas/data/lgr_j_neg-4.tsv};
\addplot [dashed, color=ForestGreen, line width=1.5pt, forget plot]
  table[]{Diagramas/data/lgr_j_neg-5.tsv};
\addplot [dashed, color=RoyalBlue, line width=1.5pt, forget plot]
  table[]{Diagramas/data/lgr_j_neg-6.tsv};
\addplot [dashed, color=red, line width=1.5pt]
  table[]{Diagramas/data/lgr_j_neg-7.tsv};
\addlegendentry{$\mora{k_{c}} < 0$}

\addplot [color=Peach, line width=1.5pt, forget plot]
  table[]{Diagramas/data/lgr_j_pos-3.tsv};
\addplot [color=cyan, line width=1.5pt, forget plot]
  table[]{Diagramas/data/lgr_j_pos-4.tsv};
\addplot [color=red, line width=1.5pt]
  table[]{Diagramas/data/lgr_j_pos-5.tsv};
\addplot [color=ForestGreen, line width=1.5pt, forget plot]
  table[]{Diagramas/data/lgr_j_pos-6.tsv};
\addplot [color=RoyalBlue, line width=1.5pt, forget plot]
  table[]{Diagramas/data/lgr_j_pos-7.tsv};
\addlegendentry{$\mora{k_{c}} > 0$}

\addplot[only marks, mark=o, mark options={}, mark size=2.0pt, draw=Fuchsia, thick, forget plot] table[]{Diagramas/data/lgr_j_neg-8.tsv};
\addplot[only marks, mark=x, mark options={}, mark size=2.5pt, draw=Fuchsia, thick, forget plot] table[]{Diagramas/data/lgr_j_neg-9.tsv};

\addplot[only marks, mark=square, mark size=2.0pt, thick, draw=Gray] table[row sep=crcr]{%
  0 0\\
  1 0\\
  0.7386 0\\
  -0.1745 -0.8588\\
  -0.1745 0.8588\\
  };
\addlegendentry{Polos en L.D. para $\mora{k_{c}}=4\unit{m}$}

\addplot[only marks, mark=square, mark size=2.0pt, thick, draw=Lavender] table[row sep=crcr]{%
  0.15 0\\
  0.8598 -0.3415\\
  0.8598 0.3415\\
  -0.24 -0.791\\
  -0.24 0.791\\
  };
\addlegendentry{Polos en L.C. para $\mora{k_{c}}=4\unit{m}$}
\end{axis}
\end{tikzpicture}%

  \caption{Lugar geométrico de las raíces para el sistema. Controlador proporcional.}
  \label{fig:lgr-j1}
\end{figure}

\FloatBarrier
\subsubsection{F}%después lo borramos
Nuevamente nos piden calcular cuando por lo menos se una raíz marginalmente estable. donde como en el \hyperref[pregunta-i]{\texttt{i)}} lo realizamos en \textit{MATLAB}, con el comando \verb|allmargin|  y  \verb|.GainMargin| ,  Donde se ocupo un un código parecido al del \autoref{lst:Cod_problema_If}. solo que se le añadió el integrador \(1 /(z-1))\) . Ademas sus raíces correspondientes en L.C las calculamos con el mismo código en \autoref{lst:Cod_problema_Jc} solo cambiando el \(k_c\) correspondiente y tomando en cuenta el L.C tal que los \( \mora{k_c}\) y sus raíces serian los  serian los siguientes : 

\begin{multicols}{2}
  \begin{itemize}
      \item \(\mora{k_{c1}} = 0.0069  \) 

        \textbf{Polos}
        \begin{align}
        \lambda_{0,1} &= 0.8746 \pm 0.4826i\\
        \lambda_{2,3} &= -0.2938 \pm 0.7608i \\
        \lambda_{4} &= 0.2279 \\
        \end{align}

      \item \(\mora{k_{c2}} = 0.0519\)

      \textbf{Polos}
      \begin{align}
        \lambda_{0,1} &= 1.2145 \pm 1.2225i \\
        \lambda_{2,3} &= -0.7113 \pm 0.7030i\\
        \lambda_{4} &= 0.3832 \\
        \end{align}

    \item \(\mora{k_{c3}} = 0\)

    \textbf{Polos}
    \begin{align}
      \lambda_{0,1} &= -0.1745 \pm 0.8588i \\
      \lambda_{2} &= 0.7386  \\
      \lambda_{3} &= 0  \\
      \lambda_{4} &= 1  \\
    \end{align}
        
  \end{itemize}
  \columnbreak
  \begin{itemize}
      \item \(\mora{k_{c4}} = -0.0066\)

      \textbf{Polos}
      \begin{align}
        \lambda_{0,1} &= -0.1258 \pm 0.9920i \\
        \lambda_{2} &= 1.3217\\
        \lambda_{3} &= 0.5272  \\
        \lambda_{4} &= -0.2077\\
        \end{align}

      \item \(\mora{k_{c5}} = -1.4428\)

      \textbf{Polos}
      \begin{align}
        \lambda_{0,1} &= -1.4173 \pm 3.7587i \\
        \lambda_{2} &= 4.8171 \\
        \lambda_{3} &= -1.0000  \\
        \lambda_{4} &= 0.4070   \\
        \end{align}       
  \end{itemize}
\end{multicols}

Los ceros del LC independiente del \(\mora{k_c}\) resultaban ser $Z_{0} = 0.4062 $ y $Z_{1} = -1.0472$ 

\begin{figure}[ht]
  \centering
  % This file was created by matlab2tikz.
%
%The latest updates can be retrieved from
%  http://www.mathworks.com/matlabcentral/fileexchange/22022-matlab2tikz-matlab2tikz
%where you can also make suggestions and rate matlab2tikz.
%
\begin{tikzpicture}

\begin{axis}[%
width=5.871cm,
height=6cm,
at={(0cm,0cm)},
scale only axis,
separate axis lines,
every outer x axis line/.append style={white!15!black},
every x tick label/.append style={font=\color{white!15!black}},
every x tick/.append style={white!15!black},
xmin=-2,
xmax=2,
every outer y axis line/.append style={white!15!black},
every y tick label/.append style={font=\color{white!15!black}},
every y tick/.append style={white!15!black},
ymin=-2,
ymax=2,
xlabel={$\mathfrak{Re}$},
ylabel={$\mathfrak{Im}$},
axis background/.style={fill=white},
legend cell align=left,
legend pos=outer north east
]
\draw[dotted, draw=gray] (axis cs:0,0) circle [radius=1];
\addplot [color=gray, dotted, forget plot]
  table[]{Diagramas/data/lgr_j_neg-1.tsv};
\addplot [color=gray, dotted, forget plot]
  table[]{Diagramas/data/lgr_j_neg-2.tsv};

\addplot [dashed, color=cyan, line width=1.5pt, forget plot]
  table[]{Diagramas/data/lgr_j_neg-3.tsv};
\addplot [dashed, color=Peach, line width=1.5pt, forget plot]
  table[]{Diagramas/data/lgr_j_neg-4.tsv};
\addplot [dashed, color=ForestGreen, line width=1.5pt, forget plot]
  table[]{Diagramas/data/lgr_j_neg-5.tsv};
\addplot [dashed, color=RoyalBlue, line width=1.5pt, forget plot]
  table[]{Diagramas/data/lgr_j_neg-6.tsv};
\addplot [dashed, color=red, line width=1.5pt]
  table[]{Diagramas/data/lgr_j_neg-7.tsv};
\addlegendentry{$\mora{k_{c}} < 0$}

\addplot [color=Peach, line width=1.5pt, forget plot]
  table[]{Diagramas/data/lgr_j_pos-3.tsv};
\addplot [color=cyan, line width=1.5pt, forget plot]
  table[]{Diagramas/data/lgr_j_pos-4.tsv};
\addplot [color=red, line width=1.5pt]
  table[]{Diagramas/data/lgr_j_pos-5.tsv};
\addplot [color=ForestGreen, line width=1.5pt, forget plot]
  table[]{Diagramas/data/lgr_j_pos-6.tsv};
\addplot [color=RoyalBlue, line width=1.5pt, forget plot]
  table[]{Diagramas/data/lgr_j_pos-7.tsv};
\addlegendentry{$\mora{k_{c}} > 0$}

\addplot[only marks, mark=o, mark options={}, mark size=2.0pt, draw=Fuchsia, thick, forget plot] table[]{Diagramas/data/lgr_j_neg-8.tsv};
\addplot[only marks, mark=x, mark options={}, mark size=2.5pt, draw=Fuchsia, thick, forget plot] table[]{Diagramas/data/lgr_j_neg-9.tsv};

\addplot[only marks, mark=square, mark size=2.0pt, thick, draw=Gray] table[row sep=crcr]{%
  0.8745 -0.4826\\
  0.8745 0.4826\\
  -0.2938 -0.7608\\
  -0.2938 0.7608\\
  0.2279 0\\
  };
\addlegendentry{Polos en L.C. para $\mora{k_{c}}=6.9\unit{m}$}

\addplot[only marks, mark=square, mark size=2.0pt, thick, draw=Lavender] table[row sep=crcr]{%
  1.2145 -1.2225\\
  1.2145 1.2225\\
  -0.7113 -0.7030\\
  -0.7113 0.7030\\
  0.3832 0\\
  };
\addlegendentry{Polos en L.C. para $\mora{k_{c}}=51.9\unit{m}$}

\addplot[only marks, mark=square, mark size=2.0pt, thick, draw=Emerald] table[row sep=crcr]{%
  -0.1258 -0.992\\
  -0.1258 0.992\\
  1.3217 0\\
  0.5272 0\\
  -0.2077 0\\
  };
\addlegendentry{Polos en L.C. para $\mora{k_{c}}=-6.6\unit{m}$}

\addplot[only marks, mark=square, mark size=2.0pt, thick, draw=Dandelion] table[row sep=crcr]{%
  -1.4173 -3.7587\\
  -1.4173 3.7587\\
  4.8171 0\\
  -1 0\\
  0.4070 0\\
  };
\addlegendentry{Polos en L.C. para $\mora{k_{c}}=-1.4428$}
\end{axis}
\end{tikzpicture}%

  \caption{Lugar geométrico de las raíces para el sistema. Controlador proporcional.}
  \label{fig:lgr-j2}
\end{figure}

\FloatBarrier
\subsubsection{G}%después lo borramos
 Analizando los \(\mora{k_c}\) obtenidos y su L.G.R llegamos a la
conclusion de que el rango en donde el sistema es estable para este caso con el controlador integrador es:
\begin{equation}
  \boxed{ 0 \leq \mora{k_{c}} \leq 6.9\unit{m} }
\end{equation}

\FloatBarrier
\subsection{Comentarios}


\textbf{!AQUÍ COMENTARIOS NO OLVIDAR!}

Posibles Comentarios
\begin{itemize}
    \item Idem (c), (d), (f) y (g) comparando con el controlador con integrador tiempo continuo
\end{itemize}
