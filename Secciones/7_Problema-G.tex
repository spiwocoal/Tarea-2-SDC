\section{Pregunta \texttt{g)}}\label{pregunta-g}
\subsection{Desarrollo}

A partir de lo que se pudo observar en los ítems anteriores, compliamos entonces
la información que se puede obtener con el L.G.R. del sistema. En la \autoref{fig:lgr-g}
se puede observar el lugar geométrico de las raíces, con los polos para cada una
de las ganancias que se determinaron previamente.

\begin{figure}[ht]
  \centering
  % This file was created by matlab2tikz.
%
%The latest updates can be retrieved from
%  http://www.mathworks.com/matlabcentral/fileexchange/22022-matlab2tikz-matlab2tikz
%where you can also make suggestions and rate matlab2tikz.
%
\begin{tikzpicture}

\begin{axis}[%
width=5.958cm,
height=6cm,
at={(0cm,0cm)},
scale only axis,
separate axis lines,
every outer x axis line/.append style={white!15!black},
every x tick label/.append style={font=\color{white!15!black}},
every x tick/.append style={white!15!black},
xmin=-12,
xmax=2,
every outer y axis line/.append style={white!15!black},
every y tick label/.append style={font=\color{white!15!black}},
every y tick/.append style={white!15!black},
ymin=-30,
ymax=30,
axis background/.style={fill=white},
legend pos=outer north east
]
\addplot [color=gray, dotted, forget plot]
  table[]{Diagramas/data/lgr_c_neg-1.tsv};
\addplot [color=gray, dotted, forget plot]
  table[]{Diagramas/data/lgr_c_neg-2.tsv};

\addplot [color=red, line width=1.5pt]
  table[]{Diagramas/data/lgr_c_pos-3.tsv};
  \addlegendentry{$\mora{k_{c}} > 0$}
\addplot [color=ForestGreen, line width=1.5pt, forget plot]
  table[]{Diagramas/data/lgr_c_pos-4.tsv};
\addplot [color=cyan, line width=1.5pt, forget plot]
  table[]{Diagramas/data/lgr_c_pos-5.tsv};

\addplot [color=red, dashed, line width=1.5pt]
  table[]{Diagramas/data/lgr_c_neg-3.tsv};
  \addlegendentry{$\mora{k_{c}} < 0$}
\addplot [color=ForestGreen, dashed, line width=1.5pt, forget plot]
  table[]{Diagramas/data/lgr_c_neg-4.tsv};
\addplot [color=cyan, dashed, line width=1.5pt, forget plot]
  table[]{Diagramas/data/lgr_c_neg-5.tsv};

\addplot[only marks, mark=o, mark size=2.0pt, thick, draw=Fuchsia, forget plot] table[]{Diagramas/data/lgr_c_pos-6.tsv};
\addplot[only marks, mark=x, mark size=2.5pt, thick, draw=Fuchsia, forget plot] table[]{Diagramas/data/lgr_c_pos-7.tsv};

\addplot[only marks, mark=square, mark size=2.0pt, thick, draw=Gray] table[row sep=crcr]{%
   0       12.1792\\
   0      -12.1792\\
  -2.8345   0\\
  };
  \addlegendentry{Polos en L.C. para $\mora{k_{c}}=22.2\unit{m}$}

\addplot[only marks, mark=square, mark size=2.0pt, thick, draw=Lavender] table[row sep=crcr]{%
  -1.4173  7.2161\\
  -1.4173 -7.2161\\
   0        0\\
  };
\addlegendentry{Polos en L.C. para $\mora{k_{c}}=-8.8\unit{m}$}

\addplot[only marks, mark=square, mark size=2.0pt, thick, draw=Emerald] table[row sep=crcr]{%
  -1.3498 +7.3193\\
  -1.3498 -7.3193\\
  -0.1350 0.0000\\
  };
\addlegendentry{Polos en L.C. para $\mora{k_{c}}=-8.3\unit{m}$}

\addplot[only marks, mark=square, mark size=2.0pt, thick, draw=Dandelion] table[row sep=crcr]{%
  -0.2362 +10.6078\\
  -0.2362 -10.6078\\
  -2.3622 0.0000\\
  };
  \addlegendentry{Polos en L.C. para $\mora{k_{c}}=10.8\unit{m}$}
\end{axis}
\end{tikzpicture}%

  \caption{Lugar geométrico de las raíces para el sistema con controlador proporcional.}
  \label{fig:lgr-g}
\end{figure}

Respecto a la estabilidad del sistema, se concluye entonces que, el sistema será
estable para ganancias en el intervalo $-8.8\unit{m} \leq \mora{k_{c}} \leq 22.2\unit{m}$.
Esto se debe a que en los límites del intervalo se encuentran las ganancias que
hacen del sistema marginalmente estable, tal como se observó en la pregunta
\hyperref[pregunta-f]{\texttt{f)}}.

Por otro lado, para que el sistema sea estable \textbf{y} oscilatorio para entrada
escalón, hace falta que este se comporte como un sistema de segundo orden. Entonces,
basándonos en la información obtenida en la pregunta \hyperref[pregunta-e]{\texttt{e)}},
el sistema se comporta como un sistema de segundo orden, y es a la vez estable para
aquellas ganancias que se encuentran en el intervalo $10.8\unit{m} \leq \mora{k_{c}} \leq 22.2\unit{m}$,
donde el límite superior es la ganancia que hace del sistema marginalmente estable.


\FloatBarrier
\subsection{Comentarios}


\textbf{!AQUÍ COMENTARIOS NO OLVIDAR!}
Posibles Comentarios
\begin{itemize}
    \item Cuál rango es mayor
    \item Siempre hay un rango para k tal que el sistema es estable y uno inestable
    \item En los rangos que no es de 1er orden hay garantías que se comporta de 2do orden
\end{itemize}
