\section{Pregunta \texttt{c)}}\label{pregunta-c}
\subsection{Desarrollo}

En este apartado nos piden calcular, los polos y ceros de el lazo directo (L.D) 
y el lazo cerrado (L.C.), por lo que primero encontramos las función de 
transferencia de cada lazo tal que son las siguientes:

\begin{itemize}
    \item \textbf{L.D. :} Sabemos que es \( \nara{k_c k_a} \frac{\rojo{\psi}(s)}{\verd{v_i}(s)}\nara{k_st} \) por lo que la función de transferencia es:
    \begin{equation}
        l(s)= \nara{k_ck_a} \frac{\nara{\nara{b_{0\omega}}}\left(\nara{b_{1\psi}}s + \nara{b_{0\psi}}\right)}
        {\left(\nara{a_{1\omega}}s + \nara{a_{0\omega}}\right)\left(\nara{a_{2\psi}}s^{2} + \nara{a_{1\psi}}s + \nara{a_{0\psi}}\right)} \nara{k_st}
    \end{equation}
    Ahora bien calculamos los polos y ceros con \textit{MATLAB} con el comando \verb|pzmap|, esto se puede ver en el \autoref{lst:Cod_problema_C}: dando los siguientes resultados:

\begin{multicols}{2}
    \textbf{Polos}
    \begin{itemize}
        \item \(P_{0,1} = -0.6597 \pm 8.8564i\) 
        \item \(P_2 = -1.5152 \)
    \end{itemize}
    \columnbreak
    \textbf{Cero}
    \begin{itemize}
        \item \(C_0 = -4.4609\)
    \end{itemize}
  \end{multicols}

    \item \textbf{L.C. :} El lazo cerrado lo sacamos simbólicamente del item \hyperref[pregunta-a]{\texttt{a)} } y luego la simplificamos:
    \begin{align*}
        &h(s) = \dfrac{\nara{k_ck_a} \frac{\nara{\nara{b_{0\omega}}}\left(\nara{b_{1\psi}}s + \nara{b_{0\psi}}\right)}
        {\left(\nara{a_{1\omega}}s + \nara{a_{0\omega}}\right)\left(\nara{a_{2\psi}}s^{2} + \nara{a_{1\psi}}s + \nara{a_{0\psi}}\right)}}{1+\nara{k_ck_a} \frac{\nara{\nara{b_{0\omega}}}\left(\nara{b_{1\psi}}s + \nara{b_{0\psi}}\right)}
        {\left(\nara{a_{1\omega}}s + \nara{a_{0\omega}}\right)\left(\nara{a_{2\psi}}s^{2} + \nara{a_{1\psi}}s + \nara{a_{0\psi}}\right)}\nara{k_{st}}} \\
        &h(s) = \dfrac{     \nara{k_ck_a}    \nara{\nara{b_{0\omega}}}\left(\nara{b_{1\psi}}s + \nara{b_{0\psi}}\right) }  {     \left(\nara{a_{1\omega}}s + \nara{a_{0\omega}}\right)\left(\nara{a_{2\psi}}s^{2} + \nara{a_{1\psi}}s + \nara{a_{0\psi}}\right)    + \nara{k_ck_a}     \nara{\nara{b_{0\omega}}}\left(\nara{b_{1\psi}}s + \nara{b_{0\psi}}\right)   
       \nara{k_{st}}}
    \end{align*}
    Al igual que en el L.D, el mismo código (\autoref{lst:Cod_problema_C}) nos entrega los polos y ceros de el L.C., estos serian los siguientes:
    \begin{multicols}{2}
        \textbf{Polos}
        \begin{itemize}
            \item \(P_{0,1} = -0.2591 \pm 10.4850i\) 
            \item \(P_2 = -2.3164 \)
        \end{itemize}
        \columnbreak
        \textbf{Cero}
        \begin{itemize}
            \item \(C_0 = -4.4609\)
        \end{itemize}
      \end{multicols}
\end{itemize}





\FloatBarrier
\subsection{Comentarios}


\textbf{!AQUÍ COMENTARIOS NO OLVIDAR!}

Posibles Comentarios
\begin{itemize}
    \item Relación entre los polos de la F. de T. en L.D. y en L.C.
    \item Relación entre los ceros de la F. de T. en L.D. y en L.C.
    \item Depende de la ganancia del controlador la ubicación de los ceros
    \item Depende de la ganancia del controlador la ubicación de los polos
    \item Cómo cambian los resultados para otro punto de operación
    \item Qué parámetro no influye en la ubicación de los polos y ceros
\end{itemize}