\section{Pregunta \texttt{c)}}\label{pregunta-c}
\subsection{Desarrollo}

En este apartado nos piden calcular, los polos y ceros de el lazo directo (L.D) 
y el lazo cerrado (L.C.), por lo que primero encontramos las función de 
transferencia de cada lazo tal que son las siguientes:

\begin{itemize}
    \item \textbf{L.D.:} Sabemos que es \( \nara{k_c k_a} \frac{\rojo{\psi}(s)}{\verd{v_i}(s)}\nara{k_st} \)
    \begin{equation}
        l(s)= \nara{k_ck_a} \frac{\nara{\nara{b_{0\omega}}}\left(\nara{b_{1\psi}}s + \nara{b_{0\psi}}\right)}
        {\left(\nara{a_{1\omega}}s + \nara{a_{0\omega}}\right)\left(\nara{a_{2\psi}}s^{2} + \nara{a_{1\psi}}s + \nara{a_{0\psi}}\right)} \nara{k_st}
    \end{equation}
\end{itemize}

\begin{itemize}
    \item \textbf{L.C.:} El lazo cerrado lo sacamos simbólicamente del item \hyperref[pregunta-a]{\texttt{a)}}
    \begin{equation}
        h(s)= \dfrac{\nara{k_ck_a} \frac{\nara{\nara{b_{0\omega}}}\left(\nara{b_{1\psi}}s + \nara{b_{0\psi}}\right)}
        {\left(\nara{a_{1\omega}}s + \nara{a_{0\omega}}\right)\left(\nara{a_{2\psi}}s^{2} + \nara{a_{1\psi}}s + \nara{a_{0\psi}}\right)}}{1+\nara{k_ck_a} \frac{\nara{\nara{b_{0\omega}}}\left(\nara{b_{1\psi}}s + \nara{b_{0\psi}}\right)}
        {\left(\nara{a_{1\omega}}s + \nara{a_{0\omega}}\right)\left(\nara{a_{2\psi}}s^{2} + \nara{a_{1\psi}}s + \nara{a_{0\psi}}\right)}\nara{k_{st}}}
    \end{equation}
\end{itemize}



\FloatBarrier
\subsection{Comentarios}


\textbf{!AQUÍ COMENTARIOS NO OLVIDAR!}

Posibles Comentarios
\begin{itemize}
    \item Relación entre los polos de la F. de T. en L.D. y en L.C.
    \item Relación entre los ceros de la F. de T. en L.D. y en L.C.
    \item Depende de la ganancia del controlador la ubicación de los ceros
    \item Depende de la ganancia del controlador la ubicación de los polos
    \item Cómo cambian los resultados para otro punto de operación
    \item Qué parámetro no influye en la ubicación de los polos y ceros
\end{itemize}