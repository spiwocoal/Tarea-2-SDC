\section{Pregunta \texttt{d)}}\label{pregunta-d}

\subsection{Desarrollo}

Se desea ahora graficar el lugar geométrico de las raíces, o L.G.R. Para esto,
se hizo uso de \textit{MATLAB} y, al igual que como se lleva haciendo hasta
ahora, de la representación del sistema que se obtiene utilizando el comando
\verb|ss|. El \textit{Control Systems Toolbox} incluye el comando \verb|rlocus|
el cual permite obtener la gráfica del L.G.R. de un sistema dado. Entonces,
se utiliza este comando para obtener una gráfica, para valores de $\mora{k_{c}}$
entre $0$ y $1$, y luego entre $-1$ y $0$ (\autoref{lst:lgr-d}).

\begin{figure}[ht]
  \centering
  % This file was created by matlab2tikz.
%
%The latest updates can be retrieved from
%  http://www.mathworks.com/matlabcentral/fileexchange/22022-matlab2tikz-matlab2tikz
%where you can also make suggestions and rate matlab2tikz.
%
\begin{tikzpicture}

\begin{axis}[%
width=5.958cm,
height=6cm,
at={(0cm,0cm)},
scale only axis,
separate axis lines,
every outer x axis line/.append style={white!15!black},
every x tick label/.append style={font=\color{white!15!black}},
every x tick/.append style={white!15!black},
xmin=-12,
xmax=2,
every outer y axis line/.append style={white!15!black},
every y tick label/.append style={font=\color{white!15!black}},
every y tick/.append style={white!15!black},
ymin=-30,
ymax=30,
axis background/.style={fill=white},
legend pos=outer north east
]
\addplot [color=gray, dotted, forget plot]
  table[]{Diagramas/data/lgr_c_neg-1.tsv};
\addplot [color=gray, dotted, forget plot]
  table[]{Diagramas/data/lgr_c_neg-2.tsv};

\addplot [color=red, line width=1.5pt]
  table[]{Diagramas/data/lgr_c_pos-3.tsv};
  \addlegendentry{$\mora{k_{c}} > 0$}
\addplot [color=ForestGreen, line width=1.5pt, forget plot]
  table[]{Diagramas/data/lgr_c_pos-4.tsv};
\addplot [color=cyan, line width=1.5pt, forget plot]
  table[]{Diagramas/data/lgr_c_pos-5.tsv};

\addplot [color=red, dashed, line width=1.5pt]
  table[]{Diagramas/data/lgr_c_neg-3.tsv};
  \addlegendentry{$\mora{k_{c}} < 0$}
\addplot [color=ForestGreen, dashed, line width=1.5pt, forget plot]
  table[]{Diagramas/data/lgr_c_neg-4.tsv};
\addplot [color=cyan, dashed, line width=1.5pt, forget plot]
  table[]{Diagramas/data/lgr_c_neg-5.tsv};

\addplot[only marks, mark=o, mark size=2.0pt, thick, draw=Fuchsia, forget plot] table[]{Diagramas/data/lgr_c_pos-6.tsv};
\addplot[only marks, mark=x, mark size=2.5pt, thick, draw=Fuchsia, forget plot] table[]{Diagramas/data/lgr_c_pos-7.tsv};

\addplot[only marks, mark=square, mark size=2.0pt, thick, draw=Gray] table[row sep=crcr]{%
  -0.6597 8.8564\\
  -0.6597 -8.8564\\
  -1.5152 0\\
  };
  \addlegendentry{Polos en L.D. para $\mora{k_{c}}=10\unit{m}$}

\addplot[only marks, mark=square, mark size=2.0pt, thick, draw=Lavender] table[row sep=crcr]{%
  -0.2591 10.485\\
  -0.2591 -10.485\\
  -2.3164 0\\
  };
  \addlegendentry{Polos en L.C. para $\mora{k_{c}}=10\unit{m}$}
\end{axis}
\end{tikzpicture}%

  \caption{Lugar geométrico de las raíces para el sistema con controlador proporcional.}
  \label{fig:lgr-d}
\end{figure}

Estos rangos se determinaron observando el L.G.R. generado sin entregarle valores
de ganancias al comando, y observando cuál sería un rango útil.

Igualmente, se graficó sobre el L.G.R. los polos del sistema en L.D. y L.C. para
una ganancia de controlador $\mora{k_{c}} = 10\unit{m}$. Los ceros de estos no
fueron graficados puesto que estos no dependen del valor que toma $\mora{k_{c}}$,
y por ende serían los mismos ceros ya presentes en la gráfica.

\subsection{Comentarios}

Se puede observar del L.G.R. obtenido (\autoref{fig:lgr-d}) que para el sistema
estudiado, el L.G.R. tiene $3$ ramas, esto debido a que el sistema es de tercer
orden y como consecuencia tendrá $3$ polos.

Otra característica de este que podemos observar es que el L.G.R. es simétrico
respecto al eje real, o eje $x$. Esta es una consecuencia de que las ramas del
L.G.R. representan la posición de los polos de un sistema. Entonces, cuando este
cuente con polos complejos, estos siempre vienen en pares conjugados. Entonces,
la posición de un polo se reflejará respecto al eje $x$, dando lugar a la simetría
observada.

Se observa además que existe un rango de ganancias para el cual el sistema es
inestable, lo cual se muestra como ramas que cruzan el eje imaginario, pasando
así al semi-plano derecho, es decir, los polos toman valores con parte real
positiva. Para este caso en concreto, esto ocurre cuando $\mora{k_{c}}$ toma
valores inferiores a $\approx -0.01$, como se puede observar en la rama azul,
o bien, cuando toma valores superiores a $\approx 0.02$, como se observa en las
ramas verdes y rojas.
