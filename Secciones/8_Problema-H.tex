\section{Pregunta \texttt{h)}}\label{pregunta-h}
\subsection{Desarrollo}

Ahora bien nos piden repetir los apartados \hyperref[pregunta-c]{\texttt{c)}}, \hyperref[pregunta-d]{\texttt{d)} },\hyperref[pregunta-e]{\texttt{e)} },\hyperref[pregunta-f]{\texttt{f)} } y \hyperref[pregunta-g]{\texttt{g)}}

\subsubsection{C} %después lo borramos
Calculamos los polos y ceros de el lazo directo (L.D) 
y el lazo cerrado (L.C.), por lo que primero encontramos las función de 
transferencia de cada lazo tal que son las siguientes:

\begin{itemize}
    \item \textbf{L.D. :} Sabemos que es \( \frac{\mora{k_c}}{s}\nara{k_a} \frac{\rojo{\psi}(s)}{\verd{v_i}(s)}\nara{k_st} \) por lo que la función de transferencia es:
    \begin{align*}
        l(s)&= \frac{\mora{k_c}}{s}\nara{k_a}\frac{\nara{\nara{b_{0\omega}}}\left(\nara{b_{1\psi}}s + \nara{b_{0\psi}}\right)}
        {\left(\nara{a_{1\omega}}s + \nara{a_{0\omega}}\right)\left(\nara{a_{2\psi}}s^{2} + \nara{a_{1\psi}}s + \nara{a_{0\psi}}\right)} \nara{k_st}\\
        l(s)&= \frac{  \mora{k_c} \nara{k_a} \nara{\nara{b_{0\omega}}}\left(\nara{b_{1\psi}}s + \nara{b_{0\psi}}\right) \nara{k_st}  }
        { s \left(\nara{a_{1\omega}}s + \nara{a_{0\omega}}\right)\left(\nara{a_{2\psi}}s^{2} + \nara{a_{1\psi}}s + \nara{a_{0\psi}} \right)   } 
    \end{align*}
    Ahora bien calculamos los polos y ceros con \textit{MATLAB} con el comando \verb|pzmap| y le añadimos al integrador al \(\mora{k_c}\), esto se puede ver en el \autoref{lst:Cod_problema_Hc}: dando los siguientes resultados:

\begin{multicols}{2}
    \textbf{Polos}
    \begin{itemize}
        \item \(P_0 = 0 \)
        \item \(P_{1,2} = -0.6597 \pm 8.8564i\) 
        \item \(P_3 = -1.5152 \)
    \end{itemize}
    \columnbreak
    \textbf{Cero}
    \begin{itemize}
        \item \(Z_0 = -4.4609\)
    \end{itemize}
  \end{multicols}



    \item \textbf{L.C. :} El lazo cerrado lo sacamos simbólicamente del item \hyperref[pregunta-a]{\texttt{a)} } y luego la simplificamos:
    \begin{align*}
        &h(s) = \dfrac{  \frac{\mora{k_c}}{s} \nara{k_a} \frac{\nara{\nara{b_{0\omega}}}\left(\nara{b_{1\psi}}s + \nara{b_{0\psi}}\right)}
        {\left(\nara{a_{1\omega}}s + \nara{a_{0\omega}}\right)\left(\nara{a_{2\psi}}s^{2} + \nara{a_{1\psi}}s + \nara{a_{0\psi}}\right)}   }{  1+  \frac{\mora{k_c}}{s} \nara{k_a} \frac{\nara{\nara{b_{0\omega}}}\left(\nara{b_{1\psi}}s + \nara{b_{0\psi}}\right)}
        {\left(\nara{a_{1\omega}}s + \nara{a_{0\omega}}\right)\left(\nara{a_{2\psi}}s^{2} + \nara{a_{1\psi}}s + \nara{a_{0\psi}}\right)}\nara{k_{st}}  } \\
        &h(s) = \dfrac{     \mora{k_c} \nara{k_a}    \nara{\nara{b_{0\omega}}}\left(\nara{b_{1\psi}}s + \nara{b_{0\psi}}\right)    }{    s \left(\nara{a_{1\omega}}s + \nara{a_{0\omega}}\right)\left(\nara{a_{2\psi}}s^{2} + \nara{a_{1\psi}}s + \nara{a_{0\psi}}\right)    + \mora{k_c} \nara{k_a}     \nara{\nara{b_{0\omega}}}\left(\nara{b_{1\psi}}s + \nara{b_{0\psi}}\right)   
       \nara{k_{st}}   }
    \end{align*}
    Al igual que en el L.D, el mismo código (\autoref{lst:Cod_problema_Hc}) nos entrega los polos y ceros de el L.C., estos serian los siguientes:
    \begin{multicols}{2}
        \textbf{Polos}
        \begin{itemize}
            \item \(P_{0,1} = -0.4593 \pm 8.7836i\) 
            \item \(P_{2,3} = -0.9580 \pm 0.9117i\) 
        \end{itemize}
        \columnbreak
        \textbf{Cero}
        \begin{itemize}
            \item \(Z_0 = -4.4609\)
        \end{itemize}
      \end{multicols}
\end{itemize}


\begin{figure}[ht]
  \centering
  \input{Diagramas/lgr_h.tex}
  \caption{Lugar geométrico de las raíces para el sistema. Controlador proporcional.}
  \label{fig:lgr-h}
\end{figure}

\FloatBarrier
\subsubsection{D}%después lo borramos

\FloatBarrier
\subsubsection{E}%después lo borramos

\FloatBarrier
\subsubsection{F}%después lo borramos

\FloatBarrier
\subsubsection{G}%después lo borramos

\FloatBarrier
\subsection{Comentarios}


\textbf{!AQUÍ COMENTARIOS NO OLVIDAR!}
Posibles Comentarios
\begin{itemize}
    \item Idem (c), (d), (e), (f) y (g) comparando con el controlador de ganancia
\end{itemize}
